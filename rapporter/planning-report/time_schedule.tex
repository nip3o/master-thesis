% [Detaljerad tidsplan som vecka för vecka beskriver examensarbetets
% [aktiviteter och milstolpar. Planerat datum för framläggning ska ingå.
% [För exjobb på avancerad nivå ska även planerat datum för
% [halvtidskontroll ingå. För dessa exjobb ska det också tydligt framgå
% [vilka resultat som förväntas ha uppnåtts vid halvtidskontrollen.]

% 1 Egenkontroll av behörighetskrav (student)
% 2 Anmälan av examensarbete i WExUpp och tilldelning av examinator och IDA-handledare (student)
% 3 Godkännande av påbörjande av examensarbete (student + examinator + studievägledare)
% 4 Möte med examinator och handledare (student)
% 5 Planeringsrapport (student)
% 6 Handledarträffar och examinatorsträffar (student)
% 7 Auskultation (student)
% 8 Halvtidskontroll (student + examinator)
% 9 Utkast till slutrapport (student)
% 10 Granskning av utkast till slutrapport (handledare)
% 11 Korrektur (student)
% 12 Granskning av slutrapport (examinator)
% 13 Korrektur (student)
% 14 Hantering av opponering (student + opponent + examinator)
% 15 Godkännande för framläggning (examinator + student)
% 16 Val av tid för framläggning (student + examinator + opponent)
% 17 Administrativ förberedelse av framläggningen (student)
% 18 Framläggning (student)
% 19 Korrigering (student)
% 20 Godkännande (examinator)
% 21 Publicering av rapport hos LiU E-Press och inlämning av reflektionsuppgift (student)
% 22 Sök priser i Sverige och utomlands (student + examinator)
% 23 Utvärdering (handledare + student + examinator)

\subsection{Week planning\footnote{Week numbers refers to the ISO 8601 week number of year}}

\subsubsection{Week 5}

This week is focused on getting to know the company and the existing web
application, to understand the basics of project planning and which role
staff manning has in this context. The project planning report should be
started, a draft of the problem problem formulation should be done.
Meeting with customers in order to understand what the product designed
during the case study should look like.


\subsubsection{Week 6}

The last parts of planning takes place and the planning report is
completed. The literature study begins. Focus on getting a good basic
knowledge of what testing is and how it can be performed in Rails.
Getting better knowledge of the involved frameworks.


\subsubsection{Week 7-14}

Literature study combined with test implementations, trying to determine
quality factors of testing and ways to automate parts of the test suite
based on other parts. Write drafts of relevant parts for the final
report. Planning design of new module to be used in evaluation study.


\subsubsection{Half-time supervision}

A suggested preliminary date for the half-time supervision is 2014-04-03.\\

The following goals should be completed at this supervision:
\begin{itemize}
  \item Something
\end{itemize}


\subsubsection{Week 15-22}

Study how the practices results could be applied to existing code and
how it could be used when extending the code with a new module. Writing
report, hand in drafts to supervisor, correct reports (iterate as
needed).


\subsubsection{Week 22-23}

Opposition of other thesis. Preparations for presentation.


\subsubsection{Week 23-24}

Final thesis presentation. Feedback corrections on report from opponent.
Post-study and evaluation report.
