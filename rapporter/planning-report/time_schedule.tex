% [Detaljerad tidsplan som vecka för vecka beskriver examensarbetets
% [aktiviteter och milstolpar. Planerat datum för framläggning ska ingå.
% [För exjobb på avancerad nivå ska även planerat datum för
% [halvtidskontroll ingå. För dessa exjobb ska det också tydligt framgå
% [vilka resultat som förväntas ha uppnåtts vid halvtidskontrollen.]

% 1 Egenkontroll av behörighetskrav (student)
% 2 Anmälan av examensarbete i WExUpp och tilldelning av examinator och IDA-handledare (student)
% 3 Godkännande av påbörjande av examensarbete (student + examinator + studievägledare)
% 4 Möte med examinator och handledare (student)
% 5 Planeringsrapport (student)
% 6 Handledarträffar och examinatorsträffar (student)
% 7 Auskultation (student)
% 8 Halvtidskontroll (student + examinator)
% 9 Utkast till slutrapport (student)
% 10 Granskning av utkast till slutrapport (handledare)
% 11 Korrektur (student)
% 12 Granskning av slutrapport (examinator)
% 13 Korrektur (student)
% 14 Hantering av opponering (student + opponent + examinator)
% 15 Godkännande för framläggning (examinator + student)
% 16 Val av tid för framläggning (student + examinator + opponent)
% 17 Administrativ förberedelse av framläggningen (student)
% 18 Framläggning (student)
% 19 Korrigering (student)
% 20 Godkännande (examinator)
% 21 Publicering av rapport hos LiU E-Press och inlämning av reflektionsuppgift (student)
% 22 Sök priser i Sverige och utomlands (student + examinator)
% 23 Utvärdering (handledare + student + examinator)

\subsection{Week planning}

\subsubsection{Week 5}

This week is focused on getting to know the company and the existing web
application, to understand the basics of project planning and which role
staff manning has in this context. The project planning report should be
started. A draft of the problem problem formulation should be done.


\subsubsection{Week 6}

The last parts of planning takes place and the planning report is
completed at the end of the week. Meeting with customers in order to
understand the problem, how it is solved today, and get their view on
what the final product should contain. Begin literature study.


\subsubsection{Week 7-}

Haxx haxx.

\subsubsection{Week 15}

Halvtidskontroll.


\subsection{Half-time report}
The preliminary date for the half-time supervision is 2014-04-10.


\subsubsection{Goals for the half-time report}

\begin{itemize}
  \item To have completed something
\end{itemize}
