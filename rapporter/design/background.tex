GOLI används i nuläget av verksamhetschefer, eller i vissa fall chefer
för ett visst röntgen-laboratorium, för att göra och följa upp
produktionsplaner. Totala produktionsbehovet räknas fram manuellt
utifrån faktorer som t.ex. antal kvinnor i länet i fallet med
mammografi, samt statistik från föregående år. Detta ligger sedan till
grund för hur många timmar som man ska bemanna med per dag, vilket i sin
tur används vid schemaläggningen - d.v.s. när de planerade timmarna ska
fördelas på olika personer.\\

I planeringsprocessen är man intresserad av hur en förändring av de
planerade aktiviteterna under året påverkar mängden timmar, så att man
kan höja produktionen för året utan att behöva schemalägga fler timmar.
Ett exempel kan vara att uppstartstiden för varje arbetspass är
konstant, vilket gör att man kan uppnå högre produktion genom att ha
några få långa pass istället för många korta pass. För att kunna
experimentera med olika sådana planer krävs att man kan tilldela ett
visst antal timmar till varje aktivitet, samt få dessa summerade.\\

Eftersom att det planerade antalet bemannade timmar per dag ligger till
grund för schemaläggning, vore det även önskvärt om denna planering gav
ett bra underlag vid schemaläggningen. Man skulle på sikt även vilja
utöka stödet för bemanning för att kunna ge en mer detaljerad bild av
vilken personal som behövs. Vissa undersökningar kräver olika typer av
kompetenser, vilket innehas av olika typer av personal. Detta problem
kan dock i realiteten vara mycket komplext och ryms inte inom
ramen för examensarbetet, men bör beaktas vid design av den
grundläggande funktionaliteten för bemanning.\\
