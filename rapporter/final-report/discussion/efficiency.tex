
\subsection{Usefulness of test coverage}

As seen in \fref{sec:results_coverage}, the test coverage for the
existing RSpec integration-tests was 42 \% before this thesis project,
which is quite much considering that the existing tests was partially
duplicated and only was focused on a few specific parts of the system.
One reason for this is found when the coverage reports are inspected,
where we can see that certain kinds code gets full statement coverage,
even though there are not any tests for them. Method definitions and
field definitions on model objects are such types, since they are
executed as soon the application loads.\\

A major part of the modules with high test coverage score in the
beginning of the case study was modules without any methods, such as
data models. Since these modules also can have methods which needs
tests, it is not possible to just exclude all such modules from the
coverage report either. This is an symptom of the weakness of statement
coverage, which is also previously discussed in
\fref{sec:theory_statement_coverage}. Statement coverage is a very weak
metric and it is absolutely possible to create a non-empty class with
100 \% statement coverage without writing any tests for it.\\

Based on the weaknesses of statement test coverage, one may thus impeach
its usefulness. We did however find that apart from the coverage score
being a bad indicator on how well-tested the code was, analyzing the
statement coverage report was highly usable for finding untested parts
of the code. The nature of our code was such that it in general
contained few branches, which of course makes statement coverage
considerably much more useful than for code with higher complexity.\\

On the client-side, we also had the ability to evaluate branch coverage.
The most important difference between the coverage scores for branch
coverage versus statement coverage is that the branch coverage is zero
for all untested functions, which makes the branch coverage score a more
sound metric for the overall testing level. Branch coverage of course
also made it possible to discover a few more untested code paths.\\


\subsection{Usefulness of mutation testing}

We believe that mutation testing may be a good alternative to code
coverage as method for evaluating test efficiency. Our small manual test
indicated that mutation testing and test coverage is related, since
several of the alive mutants modified the same line, which also showed
to have zero statement test coverage. The mutation tests also found non-
equivalent modifications to the code which was not discovered by the
tests, even though they had full statement and branch test coverage.\\

Our experiences with mutation testing however shows that more work is
required before it is possible to use it in our application. Mutation
tests may also have limited usability on the server side, which in our
case is mostly tested with higher-level integration-tests rather than
with unit-tests. As mentioned in \fref{sec:theory_mutation}, efficient
mutation testing requires the scope to be narrow and the test suite to
be fast, due to the large amount of possible mutants. This is not the
case for integration-level tests. Mutation testing might thus be more
useful for the client-side, which contains more logic and therefore is
tested with isolated unit-tests to a larger extent.\\


\subsection{Efficiency of written tests}

Since most of the implemented code consisted of extensions or
refactoring to existing files, it is hard to get a fair metric value for
how well tested the modified parts of the software is. This is because
these metrics are given in percent per file, and since the same file
contains a lot of unmodified code, that will affect the result. It is
also hard to detect which parts of the code that has been modified. We
ended up using a self-constructed, subjective metric, which imposes
questions on whether or not we can draw any solutions about how
efficient the tests for the new functionality is.\\

We can however see that the overall test coverage has increased during
the different parts of the case study, which would not be the case if
the test coverage of the newly implemented functionality was low, since
the total SLOC\footnote{Source lines of code} count has increased. Based
on this fact, combined with the results of our subjective metric, we
would argue that the efficiency of the tests written during this thesis
is high in general. Mutation testing could be used in future in order to
evaluate this conclusion further. It is however hard to tell whether or
not the reason for this is because TDD methodology has been used.\\

