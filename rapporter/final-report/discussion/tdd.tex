
We have striven for using test- and behavioral-driven development
methodologies in as strict way as possible throughout this entire
thesis. The purpose of this has been to gain experience of different
advantages and drawbacks of these methodologies. In situations where we
have found these methodologies to be unsuitable, we have however chosen
to disregard some principles of these methodologies rather than try to
apply them anyway.\\

Using a test-driven approach has is in our opinion two major advantages.
The most important is perhaps that it mitigates a sense of fear about
whether or not the implementation really works as it is supposed to, and
a fear of that things may break when code is refactored. The other
important aspect is that the written code is self-testing; you can run a
command in order to find out whether or not the code is in a working
state, rather than just test manually and see what happens.\\

Another benefit from using the test-first principle is that one often
think through the design before implementation. We have to write the
implementation in a way such that it is testable, which makes us
discover new ways of solving the problem and design our software better.
We also usually find it tedious to write tests when the implementation
is done, and we think that writing the tests first simply makes the
development more fun than writing tests after implementation.\\

It is however no axiomatic guarantee for that a more testable design is
better than a less testable design. For example, making it possible to
test a module may require it to be split up in several non-intuitive
modules, or to use a large amount of mocking. We believe that it is better
to test on a higher level and perhaps discard the test-first principle
in cases where making the code testable introduces more drawbacks than
advantages to the overall design.\\
