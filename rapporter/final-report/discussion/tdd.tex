
We have striven for using test- and behavioral-driven development
methodologies in as strict way as possible throughout this entire
thesis. The purpose of this has been to gain experience of different
advantages and drawbacks of these methodologies. In situations where we
have found these methodologies to be unsuitable, we have however chosen
to disregard some principles of these methodologies rather than try to
apply them anyway.\\

Using a test-driven approach has is in our opinion two major advantages.
The most important is perhaps that it mitigates a sense of fear about
whether or not the implementation really works as it is supposed to, and
a fear of that things may break when code is re-factored. The other
important aspect is that the written code is self-testing; you can run a
command in order to find out whether or not the code is in a working
state, rather than testing it manually and see what happens. In
particular we found writing a failing test before fixing a software
defect to be especially helpful. We could instantly see when the defect
was fixed and therefore know when we were done, and we did not have to
constantly test the software manually during development.\\

Another benefit from using the test-first principle is that one often
think through the design before implementation. We have to write the
implementation in a way such that it is testable, which makes us
discover new ways of solving the problem and design our software better.
In some cases this also reduces the amount of written code since we
strive to minimize the number of cases that we have to test. Personally
we also usually find it tedious to write tests when the implementation
is done, and our opinion is that writing the tests first makes the
development more fun than writing tests after implementation.\\

There is however no guarantee that a more testable design is better than
a less testable design. For example, making it possible to test a module
may require it to be split up in several non-intuitive modules, or to
use a large amount of mocking. We believe that it is better to test on a
higher level and perhaps discard the test-first principle in cases where
making the code testable introduces more drawbacks than advantages to
the overall design.\\

Our biggest difficulties while using TDD was the \emph{refactor}-part of
the \emph{red/green/refactor} mantra. We found it hard to know whether
or not code needed to be refactored, and to know how large part of the
code to refactor. Some small changes required large parts of the code to
be refactored in order for the code to be possible to unit-test, and in
some situations we found that the overhead of doing refactoring simply
was too large. One reason for this may be that the old implementations
has been written without testability or test-driven development in mind,
and the need for heavy refactoring would probably decrease over time if
test-driven methodologies are used.\\

Our experiences from using behavior-driven development principles is
generally positive. We found that using descriptive strings for tests
(rather than function names) forced us split up tests into smaller
parts, since each part must be possible to describe with a single
sentence. It also felt natural to share preconditions (such as \emph{if
the user is logged in}) for a set of tests using a string. One drawback
is however that the full string for a test, including preconditions,
tends to be very long. This can make it harder to perceive which of the
tests that failed.\\

While using single descriptive strings for describing tests and
preconditions in general worked very well, using an ubiquitous language
when writing and reading tests felt unnatural. We experienced that the
resulting test stories was hard to read and understand, and we could not
see any gain from using it in this particular project. It might be worth
considering for large software development projects with many roles and
multiple stakeholders, but hardly for projects where developers are in
charge of feature specifications as well as testing and
implementation.\\
