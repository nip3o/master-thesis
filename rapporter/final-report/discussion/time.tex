
\subsection{Development of test execution time during the project}

As seen in \fref{sec:results_time}, the execution time of the RSpec
integration- and unit tests has decreased for every part of the case
study, even though the number of tests has increased. On average, each
test is more than twice as fast after the second part compared to before
the case study.\\

The major reason for the increase of speed comes from the refactoring of
old tests, initiated in the first part of the case study. Some old tests
were also refactored during the second part, which explains the decrease
in test execution time between the first and the second part of the case
study. One reason for the drop in average execution time after the
second part was also the fact that a few more unit tests was written,
compared to before.\\

The single most important factor for speeding up tests when refactoring
the old tests was to reduce the construction of database objects. Before
the case study, all test data for all tests was created before every
test, since the database needs to be cleaned between each test in order
for one test not to affect other tests. By using factory objects, I
could instead create only the objects needed for each specific test,
instead of creating a huge amount of data that is not even used. RSpec
also caches factory objects between tests that use the same objects,
which gives some additional speed gain.\\


\subsection{Execution time for different test types}

In \fref{sec:theory_levels}, we discussed different levels of testing,
and how this affects the amount of executed code and the test execution
time. \Fref{tab:jasmine_times} shows that our isolated Jasmine unit
tests runs at a rate of over 700 tests per second, while our system-level
browser tests in \fref{tab:rspec_browser_times} runs at a rate of
less than 0.3 tests per second. I would therefore say that the principle
of writing many isolated unit tests and a few system tests seems
very reasonable unless we choose to discard the execution time
completely.\\

One other interesting thing is the large difference between the Cucumber
browser tests and the RSpec browser tests. Even though both types of
tests are in the category of system tests, RSpec browser tests are more
than four times faster than the Cucumber tests. One reason for this
might be the size of the test. As seen in \fref{tab:browser_coverage},
the Cucumber tests cover more code than the RSpec browser tests, and
therefore take longer time to execute.\\

Another reason for the large difference in execution time might be that
some Cucumber tests are redundant. For example, the Cucumber test suite
tests login functionality multiple times, while the RSpec browser test
suite only tests this functionality once. This is because a logged-in
user is required in order to execute most tests, and the Cucumber tests
does this by entering credentials into the login form for each test. The
RSpec test suite instead uses a test helper to log in the user before
each test starts, except for when testing the login form itself. Yet
another reason for the execution time difference might be the overhead
of parsing the ubiquitous language used by Cucumber.\\


\subsection{Importance of a fast test suite}

One may argue that there is no reason to have a fast test suite, and one
may also argue that a fast test suite is very important. I think the
importance of tests being fast depends on the mindset on the people
writing them, how often the tests are run and in which way. I will
discuss some of my own experiences on this topic.\\

Having a super-fast unit test suite such as our Jasmine tests is really
nice when using TDD methodology, since it feels like the whole test
suite is completed in the same moment a file is saved. However, using
RSpec integration tests, which runs in about 150 ms, also works
fine when using TDD-methodology as long as we only run tests for the
affected module. Running the whole integration test suite continuously
takes too much time, though.\\

For the RSpec tests, the startup time before tests actually are run is
considerably long, since Rails and related components are very large and
takes some time to load. It would be possible to separate parts of the
test suite from Rails and associated frameworks as mentioned in
\fref{sec:theory_time}, but I refrained from doing this since I
considered it to affect the code structure in a bad way. Using tools
like Zeus\footnote{\url{https://github.com/burke/zeus/}} can work around
this problem, but I was unable to evaluate this tool due to
incompatibilities with my Ruby version.\\

For browser tests, I believe that the most important thing is the ability to
run these in some reasonable time. In this particular case, I believe
that the RSpec browser test suite fast enough to be run before each
deploy without annoyance, while the Cucumber suite barely fulfills the
criterion of being run within \quotes{reasonable time}. A good approach
would be to eventually re-factor the Cucumber test suite into unit-,
integration- and RSpec browser tests. Running browser tests on a
separate testing server instead of locally is another alternative.\\

