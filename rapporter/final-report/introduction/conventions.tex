
Code examples are written in Ruby\footnote{\url{https://www.ruby-
lang.org}} 2.1.1 since that is the primary language of this thesis, but
with emphasis on being understandable by people without knowledge of
this language rather than using Ruby-specific tools and practices. For
example, implicit return statements are avoided since these are
nontrivial to understand for people used to languages without this
feature, such as Python or Java. Code examples typically originates from
implemented code, but names of classes and functions has been altered
for copyright, consistency and understandability reasons.\\

The built-in Ruby module \emph{Test::Unit} is used for general test code
examples in order to preserve independence of a specific testing
framework to as wide extent as possible, although other testing
frameworks are also mentioned and exemplified in the report.\\

The area of software development contains several terminologies which
are similar or exactly the same. In cases where multiple different
terminologies exists for a certain concept, we have chosen the term with
most hits on Google. The purpose of this was to choose the most widely-
used term, and the number of search results seemed like a good measure
for this. Footnotes with alternative terminologies is present where
applicable.\\

The term \emph{testing} will in this report refer to automatic software
testing with the purpose of finding software defects, unless specified
otherwise. We will also refer to the GOLI Kapacitetsplanering software
as \emph{the application}. We will use the term \emph{test-driven
development methodologies} as a collective term for both test- and
behavioral driven development (explained in \fref{sec:tdd} and
\ref{sec:bdd}).\\
