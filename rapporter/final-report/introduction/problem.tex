
The goal of this final thesis is to analyze how automated tests can be
introduced in an existing web application, in order to detect software
bugs and errors before they end up in a production environment. In order
to do this we will conduct a case study focused on how this can be done
in the GOLI production planning system. We use the results from this
case study in order to discuss how testing can be applied to dynamic
web applications in general.\\

The main research question is thus to determine how testing can be
introduced in the scope of the GOLI ECP application. The focus should
be on techniques which are relevant this specific application, i.e.
testing web applications that uses Ruby on Rails
\footnote{Ruby on Rails framework, \url{http://rubyonrails.org/}}
and KnockoutJS
\footnote{KnockoutJS framework, \url{http://knockoutjs.com/}} with a
MongoDB\footnote{MongoDB NoSQL database, \url{https://www.mongodb.org/}}
database system for data storage. We will investigate which kind of
problems that are specific to this kind of environment, and attempt to
find suitable solutions for them.\\
