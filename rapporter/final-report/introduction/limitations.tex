There exists different categories of software testing, for example
performance testing and security testing. The scope of this thesis is
regression- or quality assurance testing\footnotemark, in which the
purpose is to verify the functionality of a part of the system rather
than measuring its characteristics. We will also only cover automatic
testing, as opposed to manual testing where the execution and result
evaluation of the test is done by a human. The term \emph{testing} will
hereby refer to automatic software regression testing unless specified
otherwise.\\

Since the result of this thesis will be evaluated in specific web
application (i.e. GOLI Kapacitetsplanering), we will only cover
techniques which are relevant this specific application. In other words,
we will focus on testing web applications which uses Ruby on
Rails\footnotemark and KnockoutJS\footnotemark.\\

\footnotetext{This is sometimes also called functional testing, but we
will refrain from using that term since it has different meanings to
different people.}

\footnotetext{Ruby on Rails framework, \url{http://rubyonrails.org/}}

\footnotetext{KnockoutJS framework, \url{http://knockoutjs.com/}}

