During code refactoring or implementation of new features in software,
errors often occur in existing parts. This may have a serious impact on
the reliability of the system, thus jeopardizing user's confidence for
the system. Automatic testing is utilized to verify the functionality of
software in order to detect bugs and errors before they end up in a
production environment.\\

Starting new web application companies often means rapid product
development in order to create the product itself, while maintenance
levels are low and the quality of the application is still easy to
assure by manual testing. As the application and the number of users
grows, maintenance and bug fixing becomes an increasing part of the
development. The size of the application might make it implausible to
test in a satisfying way by manual testing.

The commissioner body of this project, GOLI, is a startup company
developing a web application for production planning called GOLI
Kapacitetsplanering. Due to requirements from customers, the company
wishes to extend the application to include new features for handling
staff manning. The current system uses automatic testing to some extent,
but these tests are cumbersome to write and takes long time to run. The
purpose of the thesis is to analyze how this application can begin using
tests in a good way whilst the application is still quite small. The
goal is to determine a solid way of implementing new features and bug
fixes in order for the product to be able to grow effortlessly.\\
