Software testing plays an important role in the process of verifying
software functionality and preventing bugs in production code. By
writing automated tests using code instead of conducting manual tests,
the amount of tedious work during the development process can be reduced
and the software quality can be improved.\\

This thesis presents the results of a conducted case study on how
automated testing can be used when implementing new functionality in a
Ruby on Rails web application. Different frameworks for automated
software testing is used as well as test-driven development methodology,
with the purpose of getting a broad perspective on the subject. We study
common issues with testing in these kinds of applications, and discuss
drawbacks and advantages of different testing approaches. We also look
into quality factors which are applicable for tests, and analyze how
these can be measured.\\
