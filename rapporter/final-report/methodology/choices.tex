
The choice of frameworks for development was mainly given by the
constituent, since the existing software was written in Ruby on Rails
with KnockoutJS and a MongoDB database.\\

For the choice of testing-related frameworks, we choose to look for
frequently used and active developed open source frameworks.
Technologies that are used by many people intuitively often has more
resources on how they are used, and also has the advantage of being more
likely to be recognized by future developers. Active development of used
frameworks is crucial, most importantly since they are likely to be
incompatible with future versions of other frameworks (such as Rails).
Another benefit is that new features and bug fixes are released.\\

The Ruby Toolbox website \footnote{\url{https://www.ruby-toolbox.com/}},
which uses information from the Github and RubyGems websites was
consulted in order to find frameworks with mentioned qualities.\\

\subsection{Ruby test frameworks}
RSpec, RSpec-mocks, Cucumber, Capybara, FactoryGirl, Timecop, site\_prism. (wip)

\subsection{Javascript test frameworks}
Jasmine, Karma. (wip)

\subsection{Test coverage}
\label{sec:coverage_frameworks}

There are multiple different ways of testing code coverage, and the
properties and conditions for each different kind of test coverage are
discoursed in \fref{sec:coverage}. However, we were unable to find any
test coverage tools for Ruby which analyzed anything else than statement
coverage, which is the weakest test coverage metric. Quite much effort
was spent on finding such tool, but without any success. Several
websites and Stack Overflow-answers indicates that no such tool exists
for Ruby at the time of this writing \cite{web:coverage_ruby19,
so:c1c2_coverage, so:c1_coverage, web:toolbox_code_metrics}. On one
hand, some of these sources are rather old and might be outdated, which
would indicate that such tool could have been created recently. On the
other hand would at least some of these sources probably have been
updated if such tool became available.\\

We ended up using the
SimpleCov\footnote{\url{https://github.com/colszowka/simplecov}} tool
for Ruby test coverage metrics. At the time of this writing, it is the
most used tool for coverage . It is also actively developed, works with
recent Ruby versions and RSpec versions, and produces pretty and easy-
to-read coverage reports in HTML.\\


\subsection{Mutation analysis}

There exists a few different tools for mutation analysis of Javascript
code. The ones we have found originates from academic research papers.
\citet{paper:mutandis} proposes a solution which has been implemented as
a tool called Mutandis
\footnote{\url{https://github.com/saltlab/mutandis/}}.
\citet{paper:ajaxmutator} presents another approach which has been
released as AjaxMutator \footnote{\url{https://github.com/knishiura-
lab/AjaxMutator}}. \citet{paper:webmujava} proposes a system-level
mutation testing approach called webMuJava.\\

Mutandis is based on website crawling tests. Although
\citeauthor{paper:mutandis} mentions that pure Javascript frameworks
have been tested using this tool, its implementation showed to be too
specific to be considered in our context. webMuJava does not seem to be
publicly available, and also seems to be too tightly integrated with a
specific back-end technique to be useful for Javascript-testing only.
