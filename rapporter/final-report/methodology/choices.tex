
The choice of frameworks for development was mainly given by the
constituent, since the existing software was written in Ruby on Rails
with KnockoutJS and a MongoDB database. The main server-side language
was thus Ruby, and the client-side code was written in CoffeeScript
\footnote{CoffeeScript is a scripting language which compiles into
Javascript.}.\\

For the choice of testing-related frameworks, we chose to look for
frequently used and active developed open source frameworks.
Technologies that are used by many people intuitively often has more
resources on how they are used, and also has the advantage of being more
likely to be recognized by future developers. Active development of used
frameworks is crucial, most importantly since they are likely to be
incompatible with future versions of other frameworks (such as Rails).
Another benefit is that new features and bug fixes are released. The
Ruby Toolbox website \footnote{\url{https://www.ruby-toolbox.com/}},
which uses information from the Github and RubyGems websites, was
consulted in order to find frameworks with mentioned qualities
