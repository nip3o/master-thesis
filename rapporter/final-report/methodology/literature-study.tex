
The literature study is based on the problem formulation, and therefore
focus on web application testing overall and how it can be automated. In
order to get a diverse and comprehensive view on these topics, multiple
different kinds of sources were consulted. As a complement to a
traditional literature study of peer-reviewed articles and books, we
have also chosen to also consider blogs and video recordings of talks
from developer conferences.\\

While blogs are neither published nor peer-reviewed, they often express
interesting thoughts and ideas, and often give readers a chance to leave
comments and discuss its contents. This might not qualify as a review
for a scientific publication, but it give readers larger possibilities
of leaving feedback on outdated information and fact errors. It also
makes it possible to discuss the subject to a larger extent and give
additional views on the subject.\\

Recordings of people speaking at developer conferences have similar
properties when it comes to their content, lack of reviewing process and
greater possibilities for discussion. One benefit is however that speakers
at such conferences tend to be experts on their subjects, which might
not be the case for a majority of all people writing blogs.\\

Blogs and talks from developer conferences have another benefit
over articles and books since they can be published instantly. The
review- and publication process for articles is long and may take
several months, and might also fail to be available in online databases
until after their embargo period has passed \cite{wiki:embargo,
pdf:publishing}. This can make it hard to publish up-to-date scientific
articles about some web development topics, since the most recent
releases of commonly used web frameworks are less than a year old
\cite{wiki:rails_versions, wiki:django_versions,
web:knockout_versions}.\\

Utilized alternative sources are mainly relied upon recognized people in
the open-source software community. One main reason for this is that
large parts of the web development community as well as the Ruby
community are pretty oriented around open-source software and agile
approaches. This is also the case for several test-driven techniques and
methodologies. Due to this, one might notice a tilt in this thesis
towards agile approaches and best practices used by the open-source
community.\\
