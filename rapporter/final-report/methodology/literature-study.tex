
The literature study was based on the problem formulation, and therefore
focuses on web application testing overall and how it can be automated.
In order to get a diverse and comprehensive view on these topics,
multiple different kinds of sources were consulted. As a
complement to a traditional literature study of peer-reviewed articles
and books, we chosen to also consider blogs and video recordings of
conference talks.\\

While blogs are not either published nor peer-reviewed, they often
express interesting thoughts and ideas, and may often give readers a
chance to leave comments and discuss its contents. This might not
qualify as a review for a scientific publication, but it also gives
greater possibilities of leaving feedback on outdated information and
is more open for discussion than traditional articles. Conference
talks has similar properties.\\

Blogs and conference talks does have another benefit over articles and
books since they can be published instantly. The review- and publication
process for articles is long and may take several months, and also might
not be available in online databases until after their embargo period
has passed \cite{wiki:embargo, pdf:publishing}. This can make it hard to
publish up-to-date scientific articles about some web development
topics, since the most recent releases of well-used frameworks are less
than a year old \cite{wiki:rails_versions, wiki:django_versions,
web:knockout_versions}.\\

Utilized alternative sources are mainly relied upon recognized people in
the open-source software community. Due to this, one might notice a skew
in this report towards agile approaches and best-practices used by the
open-source community.\\
