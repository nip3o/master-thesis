An alternative to draw conclusions from which paths of the code that is
run by a test, as done when using test coverage, is to draw conclusions
from what happens when we modify the code. The idea is that if the code
is incorrect, the test should fail. Thus, we can modify the code so it
becomes incorrect and then look at whether the test fails or not.\\

\emph{Mutation testing} is done by creating several versions of the
tested code where each version contains a slight modification. Each such
version containing a mutated version of the original source code is
called a \emph{mutant}. A mutant only differs at one location compared
to the original program, which means that each mutant should represent
exactly one bug. \cite{article:mutation, wiki:mutation}\\

\begin{lstlisting}[caption=Example of a piece of code before mutation.,
                   label=lst:mutation_before, float=t]
    def odd?(x, y)
        return (x % 2) && (y % 2)
    end
\end{lstlisting}


\begin{lstlisting}[caption=Mutated versions of code listing \ref{lst:mutation_before}.,
                   label=lst:mutation_after, float=t]
    def odd?(x, y)
        return (x % 2) && (x % 2)
    end

    def odd?(x, y)
        return (x % 2) || (y % 2)
    end

    def odd?(x, y)
        return (x % 2) && (0 % 2)
    end

    def odd?(x, y)
        return (x % 2)
    end
\end{lstlisting}

There are numerous ways of creating mutations to be used in mutants. One
could for example delete a method call or a variable, exchange an
operator for another, negate an if-statement, replace a variable with
zero or null-values, or something else. Code listing
\ref{lst:mutation_before} shows an example of a function that should
return true if both arguments are odd. Several mutated versions of this
example are shown in code listing \ref{lst:mutation_after}. The goal of
each mutation is to introduce a modification similar to a bug introduced
by a programmer. \cite{article:mutation}\\

All tests that we want to evaluate are run for the original program as
well as for each mutant. If the test results differ, the mutant is
\emph{killed}, which means that the test suite has discovered the bug.
Some mutants may however have a change that does not change the
functionality of the program. An example of this can be seen in
\ref{lst:mutation_eq}, where two variables are equal and therefore not
affected by replacing one of them with the other. This is called an
\emph{equivalent mutant}. The goal is to kill all mutants that are not
equivalent mutants. \cite{article:mutation, wiki:mutation}\\

\citet{article:mutation} presents the results of an experiment where
mutation testing was used in a web application with an automatically
generated test suite. Over 4500 mutants were generated, with a test suite
of 38 test cases. Running each test case for each mutant would require
over 170000 test runs. A large part of the program was therefore
discarded and the evaluation was focused on a specific part of the
software, which left 441 mutants. 223 of these were killed, 216 were
equivalent and 2 were not killed.\\

The article by \citeauthor{article:mutation} exemplifies two challenges
with mutation testing; a large amount of possible mutants, and a
possibly large amount of equivalent mutants. In order for mutant testing
to be efficient, the scope of testing must be narrow enough, the test
suite must be fast enough, and equivalent mutants must be possible to
detect or not be generated at all. \citeauthor{article:mutation} uses
manual evaluation to detect equivalent mutants, which is probably
impracticable in practice. \citet{article:eq_mutant} presents an
overview of multiple ways of dealing with equivalent mutants, but
concludes that even though some approaches looks promising, there is
still much work to be done in this field.\\

\begin{lstlisting}[caption=Example of a program with an equivalent mutant.,
                   label=lst:mutation_eq, float=t]
    def some_function(x)
        i = 0
        while i != 2 do
            x += 1
        end
        return x
    end

    def equivalent_mutant(x)
        i = 0
        while i < 2 do
            x += 1
        end
        return x
    end
\end{lstlisting}
