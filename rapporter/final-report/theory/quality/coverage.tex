\MakeShortVerb{\|}

\label{sec:theory_coverage}
Test coverage\footnote{Also known as \emph{code coverage}} is a measure
to describe to which degree the program source code is tested by a
certain test suite. If the test coverage is high, the program has been
more thoroughly tested and probably has a lower chance of containing
software bugs.\cite{wiki:coverage}\\

Multiple different categories of coverage criterion exist. The
following subsections are based on the chapter on test-case designs in
\citet{book:art_of_testing} unless mentioned otherwise.\\


\subsubsection{Statement coverage}
\label{sec:theory_statement_coverage}
\begin{lstlisting}[caption=A small example program for explaining different test coverage concepts.,
                   label=lst:small, float=t]
  def foo(a, b, x)
    if a > 1 and b == 0
      x = x / a
    end
    if a == 2 or x > 1
      x = x + 1
    end
  end
\end{lstlisting}

One basic requirement of the tests for a program could be that each
statement should be executed at least once. This requirement is called
full \emph{statement coverage}\footnote{Some websites, mostly in the
Ruby community, refers to this as C0 coverage}. We illustrate this by
using an example from \citet{book:art_of_testing}. For the program given
in \ref{lst:small}, we could achieve full statement coverage by setting
|a = 2|, |b = 0| and |x = 3|.\\

This requirement alone does however not verify the correctness of the
code. Maybe the first comparison |a > 1| really should be |a > 0|. Such
bugs would go unnoticed. The program also does nothing if x and a are
less than zero. If this was an error, it would also go unnoticed.\\

\citeauthor{book:art_of_testing} finds that these properties makes the
statement coverage criterion so weak that it often is useless. One could
however argue that it does fill some functionality in interpreted
programming languages. Syntax errors could go unnoticed in programs
written in such languages since they are executed line-by-line, and
syntax errors therefore does not show up until the interpreter tries to
execute the erroneous statement. Full statement coverage at least makes
sure that no syntax errors exists.\\


\subsubsection{Branch coverage}

In order to achieve full branch coverage\footnote{Sometimes also called
decision coverage and sometimes referred to as C1 coverage.}, tests must
make sure that each path in a branch statement is executed at least
once. This means for example that an if-statement that depends on a
boolean variable must be tested with both |true| and |false| as input.
Loops and switch-statements are other examples of code that typically
contains branch statements. In order to achieve this for the code in
\ref{lst:small}, we could create one test where |a = 3|, |b = 0|,
|x = 3| and another test where |a = 2|, |b = 1|, |x = 1|. Each of these
tests fulfills one of the two if-conditions each, so that all branches
are evaluated.\\

Branch coverage often implicates statement coverage, unless there are no
branches or the program has multiple entry points. There is however
still possible that we do not discover errors in our branch conditions
even with full branch coverage.\\


\subsubsection{Condition coverage}

Condition coverage\footnote{Also called predicate coverage or logical
coverage.} means that each condition for a decision in a program takes
all possible values at least once. If we have an if- statement that
depends on two boolean variables, we must make sure that each of these
variables are tested with both |true| and |false| as value. This can be
achieved in \ref{lst:small} with a combination of the input values
|a = 2|, |b = 0|, |x = 4| and |a = 1|, |b = 1|, |x = 1|.\\

One interesting this about the example above it that condition coverage
does not require more test cases than branch coverage, although the
former is often considered superior to branch coverage. Condition
coverage does however not necessarily imply branch coverage, even if
that is sometimes the case. A combination of the two conditions,
\emph{decision coverage}, can be used in order to solve make sure that
the implication holds.\\


\subsubsection{Multiple-condition coverage}

There is however still a possibility that some conditions mask other
conditions, which causes some outcomes not to be run. This problem can
be covered by the \emph{multiple-condition coverage} criterion, which
means that for each decision, all combinations of condition outcomes
must be tested.\\

For the code given in \ref{lst:small}, this requires the code to be
tested with $2^2 = 4$ combinations for each of the two decisions to be
fulfilled, eight combinations in total. This can be achieved with four
tests for this particular case. One example of variable values for
such test cases are |x = 2|, |a = 0|, |b = 4| and |x = 2|, |a = b = 1|
and |x = 1|, |a = 0|, |b = 2| and |x = a = b = 1|.\\

\citeauthor{book:art_of_testing} shows that a simple 20-statement
program consisting of a single loop with a couple of nested if-
statements can have 100 trillion different logical paths. While real
world programs might not have such extreme amounts of logical paths,
they are typically much larger and more complex than the simple example
presented in \ref{lst:small}. In other words can it is typically
impossible to achieve full multiple-condition test coverage.\\
