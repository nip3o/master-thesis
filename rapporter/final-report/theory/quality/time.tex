\label{sec:theory_time}
Performance of the developed software is often considered to be of great
importance in software development. Some people mean that the
performance of tests are just as important.\\

\citet{video:fast_slow_test} talks about the problem with depending too
much on large tests which are slow to run. He also mentions how the
execution time of a test can increase radically as the code base grows
bigger, even if the test itself is not changed. If the system is small
when the test is written, the test will run pretty fast even if it uses
a large part of the total system. As the system gets bigger, so does the
number of functions invoked by the test, thus increasing the execution
time.\\

One of the main purposes of a fast test suite is the possibility to use
test-driven software development methodologies. As discussed in
\fref{sec:tdd}, a fast response to changes is required in order to make
it practically possible to write tests in small iterations.\\

Even without using test-driven approaches, a fast test suite is
beneficial since it means that the tests can be run often. If all tests
can be run in a couple of seconds, they can easily be run every time a
source file in the system is changed. This gives the developer instant
feedback if something breaks.\\

In order to achieve fast tests, \citeauthor{video:fast_slow_test}
proposes writing a large amount of low-level unit tests which is focused
on a small testing part of the system, rather than many system tests
that integrates with large parts of the system.\\

\citet{video:fast_rails_tests} also emphasizes the importance of fast
tests, and proposes a way of achieving this in a Ruby on Rails
application. The basic idea is the same as proposed by
\citeauthor{video:fast_slow_test}, namely separating business logic so
it is independent from Rails and other frameworks. This makes it
possible to write small unit tests which only tests an isolated part of
the system, independent from any third-party classes.\\
