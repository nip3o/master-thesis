Test-Driven Development (TDD) originates from the Test First principle
in the eXtreme Programming (XP) methodology, and is said to be one of
the most controversial and influential agile practices
\cite{book:tdd_madeyski}.\\

\citet{book:tdd_madeyski} describes two types of software development
principles; Test First and Test Last. When following the Test Last
methodology, functionality is implemented in the system directly based
on user stories. When the functionality is implemented, tests are
written in order to verify the implementation. Tests are run and the
code is modified until there seems to be enough tests and all tests
passes.\\

Following the Test First methodology basically means doing these things
in reversed order. A test is written based on some part of a user story.
The functionality is implemented in order to make the test pass, and
more tests and implementation is added as needed until the user story in
completed \cite{book:tdd_madeyski}.\\

The Test First principle is a central theme in TDD.
\citet{book:tdd_beck} describes the basics of TDD in a \quotes{mantra}
called \emph{Red/green/refactor}. The color references refers to colors
used by often test runners to indicate failing or passing tests, and the
three words refers to the basic steps of TDD.

\begin{itemize}
    \item Red - a small, failing test is written.
    \item Green - functionality is implemented in order to get the test
          to pass as fast as possible.
    \item Refactor - duplications and other quick fixes introduces
          during the previous stage is removed.
\end{itemize}

According to \citeauthor{book:tdd_beck}, TDD is a way of managing fear
during programming. This fear makes you more careful, less willing to
communicate with others, and makes you avoid feedback. A more ideal
situation would be one there developers instead try to learn fast,
communicates much with others and searches out constructive feedback.\\

Some arrangements are required in order to practice TDD in an efficient
way, which are listed below.

\begin{itemize}
    \item Developers needs to write tests themselves, instead of relying
          on some test department writing all tests afterwards. It would
          simply not be practical to wait for someone else all the time.

    \item The development environment must provide fast response to changes.
          In practice this means that small code changes must compile fast,
          and tests need to run fast. Since we make a lot of small changes
          often and run the tests each time, the overhead would be overwhelming
          otherwise.

    \item Designs must consist of modules with high cohesion and loose coupling.
          It is very impractical to write tests for modules with many of
          unrelated input and output parameters.
\end{itemize}


