
During the ages of computers and software development, several software
development methodologies have been proposed. A software development
methodology defines different activities and models that can be
followed during software development. Such activities can for instance
be defining software requirements or writing software implementations,
and the methodology typically defines a process with a plan for how and
in which order the activities should be done. The waterfall model and
the V model are two classical examples of software development
methodologies. \cite{article:sw_methodologies}\\

Extreme programming (XP) is a software development methodology created
by Kent Beck, who published a book on the subject in 1999
\cite{wiki:xp}. This methodology was probably the first to propose
testing as a central part of the development process, as opposed to
being seen as a separate and distinct stage
\cite{article:sw_methodologies}. These ideas was later further
developed, and the concept of \emph{testing methodologies} was founded.
A testing methodology defines how testing should be used within the
scope of a software development methodology, and the following sections
will focus on the two most prominent ones.\\
