Behavior-Driven Development (BDD) is claimed to originate from an
article written by \citet{web:dan_north}, and is based on Test-Driven
Development \cite{wiki:bdd}. This section is based upon that article.\\

\citeauthor{web:dan_north} describes that several confusions and
misunderstandings often appeared when he taught TDD and other agile
practices in projects. Programmers had trouble to understand what to
test and what not to test, how much to test at the same time, naming
their tests and understanding why their tests failed.
\citeauthor{web:dan_north} thought that it must be possible to introduce
TDD in a way that avoids these confusions.\\

Instead of focusing on what test cases to write for a specific feature,
BDD instead focuses on behaviors that the feature should imply. Instead
of using regular function names, each test is described by a string
(typically starting with the word \emph{should}). For a function
calculating the number of days left until a given date, this could for
example be \quotes{should return 1 if tomorrow is given} or
\quotes{should raise an exception if the date is in the past}.\\

Using strings instead of function names solves the problem of naming
tests - the string describing the behavior is used instead of a
traditional function name. It also makes it possible to give a human-
readable error if the module fails to fulfill some behavior, which can
make it easier to understand why a test fails. It also sets a natural
barrier for how large the test should be, since it must be possible to
describe in one sentence.\\

After coming up with these ideas, \citeauthor{web:dan_north} realized
that writing behavior-oriented descriptions about a system had much in
common with software analysis. They came up with scenarios on the
following form to represent the purpose and preconditions for
behaviors:\\

\emph{Given} some initial context (the givens),\\
\emph{When} an event occurs,\\
\emph{Then} ensure some outcomes.\\

By using this pattern, analysts, developers and testers can all use the
same language, and is hence called a \emph{ubiquitous language}. Multiple
scenarios are written by analysts to specify the properties of the
system, which can be used by developers as functional requirements, and
as desired behaviors when writing tests.\\
