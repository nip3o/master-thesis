
\label{sec:theory_levels}

One fundamental part of all software development is the concept of
abstraction. Abstraction can be described as a way of decomposing an
application into different levels, with different level of detail. This
permits the developer to ignore certain details of the software, and
instead focus on other details.\\

Consider the development of a simple game with basic graphics. On the
lowest level possible, a such game requires a tremendous amount of work
in order to shuffle data between hardware buses, perform memory accesses
and CPU operations. By using higher abstraction levels, one can use
third-party frameworks for drawing graphics to the screen and detecting
collisions. The operating system and programming language takes care of
handling bus-accesses and memory management. This allows the developer
to focus on designing the game logic itself, rather than bothering with
drawing individual pixels or figuring out where in the memory to store
data. \cite{paper:abstraction}\\

In the same way, testing can be performed at several different levels.
There are several ways of defining these levels, but one way of
describing it is like a pyramid as seen in \fref{fig:testing_pyramid}.
We can imagine testing at different levels as holding a flashlight at
different levels of the pyramid. If we hold the flashlight at the top of
the pyramid, the flashlight will illuminate a large part of the pyramid.
If the flashlight is hold at the bottom of the pyramid, a much smaller
piece of the pyramid will be illuminated. Similar to this, testing at a
high level permit us to ignore a lot of details. Due to the high level
of abstraction, a large part of the code must be run in order for the
test to be completed. Testing at a lower level requires a much smaller
part of the code to be run. Different levels of testing have different
advantages, drawbacks and uses, which are covered in the following
subsections.\\

\begin{figure}
\centering
\includegraphics[width=0.7\textwidth]{theory/levels/triangle}
\caption{The software testing pyramid, with two flashlights at different
         levels illustrating how the level of testing affects the amount
         of tested code.}
\label{fig:testing_pyramid}
\end{figure}
