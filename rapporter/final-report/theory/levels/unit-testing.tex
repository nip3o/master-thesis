\MakeShortVerb{\|}

% In order to verify the functionality of program, it is possible to use a
% formal logical proof. This is done dividing the program into small
% pieces of logic, and the possible input data into different classes.
% \citet{book:pfleeger}

Unit testing\footnote{Also called low-level testing, module testing or
component testing} refers to testing of small parts of a software. In
practice, this often means testing a specific class or method of a
software. The purpose of unit tests is to verify the implementation,
i.e. to make sure that the tested unit works correct
\cite{book:pfleeger}.\\

Since each unit test only covers a small part of the software, it is
easy to find the cause for a failing test. Writing unit tests alone does
not give a sufficient test coverage for the whole system, since unit
tests only assures that each single tested module works as expected. A
well- tested function for validating a 12-digit personal identification
number is worth nothing if the module that uses it passes a 10-digit
number as input \cite{wiki:unittests}.\\


\subsubsection{Testability}

Writing unit tests might be really hard or really easy depending on the
implementation of the tested unit. \citet{video:misko_psychology} claims
that it is impossible to apply tests as some kind of magic after the
implementation is done. He demonstrates this by showing an example of
code written without tests in mind, and points out which parts of the
implementation that makes it hard to unit-test.
\citeauthor{video:misko_psychology} mentions global state variables,
violations of the Law of Demeter, global time references
and hard-coded dependencies as some causes for making implementations
hard to test.\\

Global states infers a requirement on the order the tests
must be run in, which is bad since it is often non-deterministic and
therefore might change between test runs. Global time references is bad
since it depends on the time when the tests are run, which means that a
test might pass if it is run today, but fail if it is run tomorrow.\\

The Law of Demeter\footnote{Also called the principle of least
knowledge} means that one unit should only have limited knowledge of
other units, and only communicate with related modules
\cite{wiki:demeter}. If this principle is not followed, it is hard to
create an isolated test for a unit which does not depend on unrelated
changes in some other module. The same thing also applies to the usage
of hard-coded dependencies. This makes the unit dependent on other
modules, and makes it impossible to replace the other unit in order to
make it easier to test.\\

\citeauthor{video:misko_psychology} shows how code with these issues can
be solved by using dependency injection and method parameters instead
of global states and hard-coded dependencies. This makes testing of the
unit much easier.\\


\subsubsection{Stubs, mocks and fakes}

Another way of dealing with dependencies on other modules is to use some
kind of object that replaces the other module. The replacement object
has a known value that is specified in the test, which means that
changes to the real object will not affect the test. The reasons for
using replacement objects is often to make it more robust to code
changes outside the tested unit. It might also be used instead of calls
to external services such as web-based API:s in order to make the tests
run when the service is unavailable, or to be able to test
implementations that depends on classes that has not been yet.\\

The naming of different kinds of replacement objects may differ, but two
often used concepts are \emph{stubs} and \emph{mocks}. Both these
replacement objects are used by the tested unit instead of some other
module, but mocks sets expectations on how it can be used by the tested
module on beforehand. \cite{web:mocks_arent_stubs}\\

Another type of replacement object are \emph{fakes} or \emph{factory
objects}. This kind of object is typically provides a real
implementation of the object that it replaces, as opposed to a stub or
mock which only has just as many methods or properties that is needed
for the test to run. The difference between a fake object and the real
object is typically that fake objects uses some shortcut which does not
work in production. One example is objects that are stored in memory
instead of in a real database, in order to gain performance.
\cite{web:mocks_arent_stubs}\\

\citet{video:boundaries} mentions some of the drawbacks with using mocks
and stubs. If the interface of the replaced unit changes, this might not
be noticed in our test. Consider the scenario given in \ref{lst:mocks}.
In this example we have written a test for the |take_cookie()| method of
the |CookieJar| class, which replaces the |eat_cookie()| method with a
stub in order to make the |CookieJar| class independent of the |Cookie|
class.\\

If we rename the |eat_cookie()| method to |eat()| without changing the
test or the implementation of |take_cookie()|, the test might still pass
although the code will fail in a production environment. This is since
we have mocked an object which no longer exists in the |Cookie| class.\\

Some testing frameworks and plug-ins detects replacement of non-existing
methods and warns or makes the test fail if such mocks are created
\cite{video:boundaries}. Another possible solution is to re-factor the
code to avoid the need for mocks or stubs.\\

\begin{lstlisting}[caption=Example of how mocking might make tests unaware
                           of changes which breaks functionality,
                   label=lst:mocks, float=t]
class Cookie
    def eat_cookie
        return "Cookie is being eaten"
    end
end

class CookieJar
    def take_cookie
        self.cookies.first.eat_cookie()
        return "One cookie in the jar was eaten"
    end
end

def test_cookie_jar
    Cookie.eat_cookie = Mock
    assert CookieJar.new.take_cookie() == "One cookie in the jar was eaten"
end
\end{lstlisting}
