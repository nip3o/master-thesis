\MakeShortVerb{\|}

% In order to verify the functionality of program, it is possible to use a
% formal logical proof. This is done dividing the program into small
% pieces of logic, and the possible input data into different classes.
% \citet{book:pfleeger}

\emph{Unit testing}\footnote{Also called low-level testing, module
testing or component testing.} refers to testing of small parts of a
software program. In practice, this often means testing a specific class
or method of a software. The purpose of unit tests is to verify the
implementation, i.e. to make sure that the tested unit works correct
\cite{book:pfleeger}. Since each unit test only covers a small part of
the software, it is easy to find the cause for a failing test. On the
other hand, a single unit test only verifies that a small part of the
application works as expected.\\

\subsubsection{Testability}

Writing unit tests might be hard or easy depending on the implementation
of the tested unit. \citet{video:misko_psychology} claims that it is
impossible to apply tests as some kind of magic after the implementation
is done. He demonstrates this by showing an example of code written
without tests in mind, and points out which parts of the implementation
that makes it hard to unit-test. \citeauthor{video:misko_psychology}
mentions global state variables, violations of the Law of
Demeter\footnote{Also called the principle of least knowledge.}, global
time references and hard-coded dependencies as some causes for making
implementations hard to test.\\

Global states infer a requirement on the order the tests must be run in.
This is bad since the order might change between test runs, which would
make the tests fail for no reason. Global time references is bad since
it depends on the time when the tests are run, which means that a test
might pass if it is run today, but fail if it is run tomorrow.\\

The Law of Demeter means that one unit should only have limited
knowledge of other units, and only communicate with related modules
\cite{wiki:demeter}. If this principle is not followed, it is hard to
create an isolated test for a unit that does not depend on unrelated
changes in some other module. The same thing also applies to the usage
of hard-coded dependencies. This makes the unit dependent on other
modules, and makes it impossible to replace the other unit in order to
make it easier to test.\\

\citeauthor{video:misko_psychology} shows how code with these issues can
be solved by using dependency injection and method parameters instead
of global states and hard-coded dependencies. This makes testing of the
unit much easier.\\


\subsubsection{Stubs, mocks and factory objects}
\label{sec:theory_mocks}

A central part of unit testing is the isolation of each unit, since we
do not want tests to fail because of unrelated changes. One way of
dealing with dependencies on other modules is to use some kind of object
that replaces the other module during the execution of the test. The
replacement object has a known value that is specified by the test,
which means that changes to the real object will not affect the result
of the test.\\

The naming of different kinds of replacement objects may differ, but two
widely used concepts are \emph{stubs} and \emph{mocks}. Both these
replacement objects are used by the tested unit instead of some other
module, but mocks also sets expectations on how it can be used by the
tested module beforehand. \cite{web:mocks_arent_stubs}\\

As mentioned, the main reason for using mocks or stubs is often to make
the test more robust to code changes outside the tested unit. Such
replacement objects might also be used instead of calls to external
services such as web- based APIs in order to make the tests run when the
service is unavailable, or to be able to test implementations that
depends on classes that has not been implemented yet.\\

Another type of replacement object is called \emph{factory
objects}\footnote{Also called fakes.}. This kind of object typically
provides a real implementation of the object that it replaces, as
opposed to a stub or mock, which only has just as many methods or
properties that is needed for the test to run. The difference between a
fake object and the real object is that fake objects may use some
shortcut that does not work in production. One example is objects that
are stored in memory instead of in a real database, in order to gain
performance. Factory objects also provide a single entry point for
constructing data, which makes it easier to maintain changes to the
object structure. \cite{web:mocks_arent_stubs}\\

\citet{video:boundaries} mentions some of the drawbacks with using mocks
and stubs. If the interface of the replaced unit changes, this might not
be noticed in our test. Consider the scenario given in \ref{lst:mocks}.
In this example we have written a test for the |take_cookie()| method of
the |CookieJar| class, which replaces the |eat_cookie()| method with a
stub in order to make the |CookieJar| class independent of the |Cookie|
class. If we rename the |eat_cookie()| method to |eat()| without
changing the test or the implementation of |take_cookie()|, the test
might still pass although the code would fail in a production
environment. This is since we have mocked an object that no longer
exists in the |Cookie| class.\\

Some testing frameworks and plug-ins detect replacement of non-existing
methods, and give a warning or make the test fail in these situations.
Another possible solution is to do refactoring of the code to avoid the
need for mocks and stubs. \cite{video:boundaries}\\

\begin{lstlisting}[caption=Example of how mocking might make tests unaware
                           of changes which breaks functionality.,
                   label=lst:mocks, float=t]
class Cookie
    def eat_cookie
        return "Cookie is being eaten"
    end
end

class CookieJar
    def take_cookie
        self.cookies.first.eat_cookie()
        return "One cookie in the jar was eaten"
    end
end

def test_cookie_jar
    Cookie.eat_cookie = Mock
    assert CookieJar.new.take_cookie() == "One cookie in the jar was eaten"
end
\end{lstlisting}
