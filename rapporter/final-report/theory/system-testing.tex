System testing is conducted on the whole, integrated software system.
Its purpose is to test if the end product fulfills specified
requirements, which includes determining whether all software units (and
hardware units, if any) are properly integrated with each other. In some
situations, parameters such as reliability, security and usability is
tested.\cite{book:adp}\\

The most relevant part of system testing for the scope of this thesis is
functional testing. The purpose of this is to verify the functional
requirements of the application at the highest level of abstraction. In
other words, one wants to make sure that the functionality used by the
end-users works as expected. This might be the first time where all
system components are tested together, and also the first time the
system is tested on multiple platforms. Because of this, some types of
software defects may never show up until system testing is
performed.\cite{book:adp}\\

\citeauthor{book:adp} proposes that system testing should be performed
as black box tests corresponding to different application use-cases. An
example could be testing the functionality of an online library catalog
by adding a new users to the system, log in and perform different types
of searches in the catalog. Domain-testing techniques such as boundary-
value testing are used in order to narrow down the number of test
cases.\\
