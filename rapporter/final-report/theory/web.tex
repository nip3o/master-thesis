\MakeShortVerb{\|}

\label{sec:testing_web}

\subsection{Typical characteristics of a web application}

Web applications typically share multiple properties with traditional
software, i.e. software that runs as an application locally on a single
computer. Many languages can be used for writing traditional software as
well as web applications, and the objective of doing software testing is
typically the same. There is however some key differences and features
exhibited by web applications. \citet{book:web} mentions the following
specific characteristics for web applications:\\

\begin{itemize}

\item It can be used by a large number of users from multiple
geographical locations at the same time.

\item The execution environment is very complex and may include
different hardware, web servers, Internet connections, operating systems
and web browsers.

\item The software itself typically consists of heterogeneous
components, which often uses several different technologies. For example
it may have a server component that uses Ruby and Ruby on Rails, and a
client component that uses HTML, CSS and Javascript.

\item Some components, such as parts of the graphical interface, may be
generated at run time depending on user input and the current state of
the application.

\end{itemize}

Each of these characteristics contributes to different challenges when
doing software testing. While some of these characteristics pose testing
issues that are outside the scope of this thesis, other issues are
highly relevant. It may for instance be impossible to use the same
testing framework for the server-side components as for the client-side
components since the components are written in different programming
languages. Another example is the complexity of the execution
environment, which may result in defects that occur in some environments
but not in others.\\


\subsection{Levels of testing}

Defining levels of testing requires greater attention in web
applications than in traditional software due to its heterogeneous
nature. For example, it is hard to define the scope of a unit during
unit testing since it depends on the type of component. Defining levels
of integration testing infer the same types of problem, since one may
for example choose to either integrate different smaller server-side
units, or test the integration between server-side and client- side
components. \cite{book:web}

\citeauthor{book:web} propose that client-side unit testing should be
done on each page, where scripting modules as well as HTML statements
and links are tested. Server-side unit testing should test storage of
data into databases, failures during execution of controllers and
generation of HTML client pages. Integration testing should be done
testing the integration of multiple pages, for example testing
redirections and submission of forms. System testing is done by testing
different use cases and by detecting defective pointing of links between
pages.\\

The book by \citeauthor{book:web} is rather old, and one could argue
that some parts of the proposed testing approach would not fit into a
modern web application. As one example, several modern web applications
consist of one single page, which makes it impossible to test the
integration of several pages, form submissions or to test link between
pages. One may also fail to see the purpose of testing hyperlinks at all
three testing levels, and would also claim that generation of such links
is handled to a large extent by modern web frameworks and thus is
outside our application scope. One could also claim that the interaction
between different pages is covered by the use cases evaluated during the
system testing.\\

We do however believe that unit testing client-side as well as server-
side code respectively is important, and also that use-cases are a
relevant approach when doing system testing.\\


\subsection{Browser testing}

As previously mentioned, a web application may be run on a variety of
different operating systems and browsers which may cause defects
specific to each environment. In order to discover defects of that kind,
one must be able to test the web application in each supported
environment.\\

The \emph{browser test} is a kind of test where a web browser is
controlled in order to simulate the behavior of a real user, by clicking
on buttons and entering text into text fields. This approach tests the
application in the same way as when conducting manual testing, which is
an advantage as well as a drawback. On one hand, this way of testing is
by far the most realistic way of testing an application. On the other
hand it often tests a lot of code and it may take a lot of effort to
test things thoroughly and to find the cause of bugs when tests fail.
Because of this, browser tests are generally only used when conducting
system testing.\\

Cross-browser compatibility is an important issue for many web
applications. This basically means that the application should work in
the same way regardless of which browser the user are using, at least as
long as we have chosen to support the web browser in question. A browser
test can often be run in multiple browsers, and can thus help finding
browser-specific software defects.\\

Browser tests are slow, mostly due to the level of testing, but also due
to the overhead of executing scripts, rendering pages and loading
images. Another downside is that it requires a full graphical desktop
environment and thus may take some effort and resources to set up on a
server. One way of achieving higher test execution speed is to use a
\emph{headless browser}. A headless browser is a web browser where most
functionality of modern browsers has been stripped away in order to gain
speed. It typically does not have any graphical interface and therefore
does not require a graphical desktop environment. \cite{web:headless}\\

One may argue that by running the tests in a headless browser is
pointless since no real users uses a such browser. Therefore, we cannot
be sure that the tested functionality actually would work in a real
browser just because it works in a headless browser. It is of course
also impossible to find browser-specific defects when using a headless
browser. It would be possible to find bugs specific to the headless
browser itself, but that would be rather pointless. \citet{web:headless}
however claims that using a headless browser covers the majority of all
cases and can be used in the beginning of projects where cross-browser
compatibility is less of an issue.\\

\citet{web:page_object} explains that browser tests often tend to be
bound to the HTML and the CSS classes of the tested page, since we need
to locate elements in order to click buttons or fill in forms. This
makes the tests fragile since changes in the user interface may break
them, and if the same elements are used in many tests, we may need to go
over large parts of our test suite.\\

One way of writing browser tests without making them bound to elements
of a page is called the \emph{page object pattern}. The basic principle
is that the elements of a page are encapsulated into one or several
\emph{page objects}, which in turn provides an API that is used by the
test itself. If we have a page with a list of items as well as a form
for creating new items, we can for instance create one page object for
the page itself, another for the form and yet another for the list of
items. The page object for the page itself provides methods for
accessing the page objects for the form and the list of items. The page
object for the form may provide methods for entering data, and the page
object for the list may provide methods for accessing each item in the
list. \cite{web:page_object}

