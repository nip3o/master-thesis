\MakeShortVerb{\|}

Software is written by humans and therefore has bugs. This quote is
claimed to have been coined by John Jacobs\cite{web:quote_jacobs}, and
explains the basic reason for software testing. Most programmers would
agree that defects (or \quotes{undesirable features}), tend to show up
during the software development process as well as in the finished
software. The ultimate goal of the testing process is to establish that
a certain level of software quality has been reached, which is achieved
by revealing defects in the code \cite{book:adp}.\\

Automated software testing is typically performed by writing pieces of
code\footnote{There are also techniques for recoding clicks when testing
graphical interfaces, but that is outside the scope of this thesis.}
called \emph{tests}. Each test uses the implemented code that we want to
evaluate, and performs different assertions to make sure that the result
is the same as we expect. Code listing \ref{lst:basic_function} shows a
very simple function and \ref{lst:dummy_test} shows a piece of code for
testing it.\cite{wiki:test_automation}\\

For writing more complicated tests, and in order to manage collections
of tests, writing a bunch of |if|-statements all the time is repetitive
and tedious. Several languages provide |assert| statements\footnote{The
Ruby core module does not provide this as a reserved word, but it is
included as part of the Test::Unit module.}, which checks if the given
condition is true, and raises an exception otherwise (the assertion
\emph{fails}). The same test using assertions is shown in code listing
\ref{lst:basic_test}. However, we often want to have a little bit more
support than just |assert| statements. We may for example want to run a
piece of code before each test for a specific module, present more
helpful error- messages depending on the compared data types. A
\emph{testing framework}\footnote{Also called testing tool.} typically
provides such functionality.\cite{wiki:test_automation}\\

The code for each test is executed by some kind of \emph{test runner}
which finds and executes our collection of tests (called \emph{test
suite}), and then collects and reports the results. The test runner is
often included as a part of the testing framework. If at least one
assertion in a test fails during the execution, the test is considered
to be failed. Test may also fail due to syntax errors and unexpected
exceptions.\\

\begin{lstlisting}[caption=An example function.,
                   label=lst:basic_function, float=t]

def plus(x, y)
  return x + y
end

\end{lstlisting}


\begin{lstlisting}[caption=A piece of code that could be used for testing the function in code listing \ref{lst:basic_function}.,
                   label=lst:dummy_test, float=t]

def test_plus
  if !(plus(1, 2) == 3)
    raise
  end
  if !(plus(3, -4) < 0)
    raise
  end
end

\end{lstlisting}

\begin{lstlisting}[caption=A basic test for the function in code listing \ref{lst:basic_function}.,
                   label=lst:basic_test, float=t]

def test_plus
  assert plus(1, 2) == 3
  assert plus(3, -4) < 0
end

\end{lstlisting}
