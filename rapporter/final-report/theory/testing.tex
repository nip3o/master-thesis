\MakeShortVerb{\|}

Software is written by humans and therefore has bugs. This quote is
claimed to have been coined by John Jacobs\cite{web:quote_jacobs}, and
explains the basic reason for software testing. Most programmers would
agree that defects (or \quotes{undesirable features}), tend to show up
during the software development process as well as in the finished
software. The ultimate goal of the testing process is to establish that
a certain level of software quality has been reached, which is achieved
by revealing defects in the code \cite{book:adp}.\\

Automated software testing is typically performed by writing pieces of
code called \emph{tests}. Each test uses the implemented code that we
want to evaluate, and performs different assertions to make sure that
the result is the same as we expect. Code listing
\ref{lst:basic_function} and \ref{lst:basic_test} shows a very simple
function and a test for it. The |assert|-statement checks if the given
condition is true, and raises an exception otherwise (the assertion
\emph{fails}).\\

The code for each test is executed by some kind of \emph{test runner},
which is often included as a part of the test framework. If at least one
assertion in a test fails during the execution, the test has failed.\\


\begin{lstlisting}[caption=An example function.,
                   label=lst:basic_function, float=t]

def plus(x, y)
  return x + y
end

\end{lstlisting}


\begin{lstlisting}[caption=A basic test for the function in code listing \ref{lst:basic_function}.,
                   label=lst:basic_test, float=t]

def test_plus
  assert plus(1, 2) == 3
  assert plus(3, -4) < 0
end

\end{lstlisting}
