
The case study will be divided into three sub parts. The purpose of each
sub part is to evaluate some aspects of the testing approach, in order
to give a good evaluation of the chosen testing approach when
combined.\\


\subsection{Refactoring of old tests}

There have been attempts to introduce testing of the ECP system a while
back, but this has stopped since the chosen approaches was found to be
very cumbersome. At the start of this project, the implemented tests had
not been maintained for a very long time, which resulted in that many
tests failed although the system itself worked fine.\\

As mentioned in section \ref{sec:hypothesis}, TDD methodology will be
used during the case study. This methodology is based on the fact that
tests are written before implementation of new features and then run
iteratively during development. The test suite should pass, then fail
after a new test has been implemented, and then pass again after the new
feature has been implemented. This of course presupposes that existing
tests can be run and give predictable results.\\

The first part of the case study is therefore to make all old tests run.
Apart from being a prerequisite for new tests and features to be
implemented, it also gives a view on how tests are affected as new
functionality is implemented. This is especially interesting since it
otherwise would be impossible to evaluate such factors in the scope of a
master thesis. It also gives a perspective on some of the advantages
and drawbacks of the old testing approach.\\

Another drawback of the old tests are the fact that they run too slow in
order to be continuously in a TDD manner. Another objective of this part
of the case study is therefore to make them faster, so at least some of
the tests can be run continuously.\\


\subsection{Implementation of new functionality}

As mentioned in section \ref{sec:background}, the commissioner body of
this project wishes to implement support for staff manning in the ECP
system. This functionality is implemented and tests are written for
new parts of the system as well as for re-factored code.\\

The purpose of this part is besides implementing the new feature itself,
to evaluate test-driven development and how tests and implementation
code can be written together. We also evaluate how different kinds of
tests serves different purposes in the development process.\\


\subsection{Increasing test coverage}

Implement more tests for old functionality.

(wip)
