
This chapter presents some final thoughts and reflections on the results
and summarizes the previously discussed conclusions.\\

\begin{itemize}

\item  Working with the chosen stack of testing frameworks (RSpec and
Jasmine with the Karma test runner) has worked very well for testing
this particular application. We believe that it would also work very
well for testing most similar web applications.\\

\item The experience of using test-driven development as well as some
topics originating from behavior-driven development has been pleasant.
While this is a subjective opinion, we believe that this might be the
case for many other software developers and for other projects as well.
We have also found that the newly implemented functionality is well
tested and that its test efficiency is high, which may be another
benefit from using these methodologies. Writing tests using an
ubiquitous language is however not beneficial for this particular
application.\\

\item The combination of many integration- and unit-tests, complemented
by a few browser tests was successful for this project. It might however
be hard to determine the best level of testing for a certain
functionality. This basically depends on the application, and it is not
possible to say that either writing higher-level unit- or integration-
test is generally either better or worse than writing lower-level unit
tests. However, we would not recommend using browser-level tests as the
primary testing method for most projects.\\

\item A test suite can speed up significantly by using factory objects
rather than using fixtures or manually create data when the test suite
is initialized. While these tests still are not in the same magnitude of
speed as the low-level unit tests written for the client side, they are
still fast enough to be usable for using test-driven development
methodologies.\\

\item Using metrics for test efficiency such as test coverage is usable
for finding parts of the code that lacks testing, in order to make tests
better. We believe that statement test coverage is usable, even if
branch coverage is more helpful and returns more fair coverage
percentages. The increased test efficiency could possibly lead to
finding more software defects due to more efficient tests, but we have
not found any such defects by using test efficiency metrics in this
project.\\

\end{itemize}
