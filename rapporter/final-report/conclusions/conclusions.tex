This chapter presents some final thoughts and reflections on the results
and summarizes the previously discussed conclusions.\\

The experience of using test-driven methodology as well as some topics
originating from behavior-driven development has been pleasant. While
this is a subjective opinion, we believe that this might be the case for
many other software developers and for other projects as well. We have
also found that the newly implemented functionality is well tested and
that its test efficiency is high, which may be another benefit from
using these methodologies.\\

The combination of many integration- and unit-tests, complemented by a
few browser tests was successful for this project. It might however be
hard to determine the best level of testing for a certain functionality.
This basically depends on the application, and it is not possible to say
that either writing higher-level unit- or integration-test is generally
either better or worse than writing lower-level unit tests. However, we
would not recommend using browser-level tests as the primary testing
method for most projects.\\

A test suite can speed up significantly by using factory objects rather
than using fixtures or manually create data when the test suite is
initialized. While these tests still are not in the same magnitude of
speed as the low- level unit tests written on the Javascript side, they
are still fast enough to be usable for using test-driven development
methodologies.\\

Using metrics for test efficiency such as test coverage is usable for
finding parts of the code that lacks testing, in order to make tests
better. We believe that statement test coverage is usable, even if
branch coverage is more helpful and returns more fair coverage
percentages. The increased test efficiency could possibly lead to
finding more software defects due to more efficient tests, but we have
not found any such defects by using test efficiency metrics in this
project.\\
