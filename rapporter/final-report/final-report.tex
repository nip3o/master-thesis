% !TEX TS-program = pdflatex
% !TEX encoding = UTF-8 Unicode

\documentclass[a4paper]{report}

\usepackage[english]{babel}
\usepackage[T1]{fontenc}
\usepackage[utf8]{inputenc}
\usepackage[pdftex]{graphicx}
\usepackage{float}
\usepackage{fancyhdr}
\usepackage[toc,page]{appendix}
\usepackage{listings}
\usepackage{booktabs} % for much better looking tables
\usepackage{array} % for better arrays (e.g. matrices) in maths
\usepackage{paralist} % very flexible & customizable lists (e.g. enumerate, etc.)
\usepackage{verbatim} % adds environment for commenting out blocks of text & for better verbatim
\usepackage{subfig} % make it possible to include more than one captioned figure / table in a single float
\usepackage[margin=2.5cm]{geometry}

\usepackage{titling} % for useful adjustments to the titles
\usepackage[hyphens]{url} % for simple URL formatting
\usepackage[numbers,sort]{natbib} % automatic short-hand citations

\usepackage[plain]{fancyref} % auto-detection of cross-referenced types
\usepackage{listings} % code listings
\usepackage{shortvrb} % shorthand pre-formatted text

\usepackage{hyperref} % useful stuff for references
\usepackage[hypcap]{caption} % smarter pointing for reference links

\usepackage[nottoc,notlof,notlot]{tocbibind} % Put the bibliography in the ToC
\usepackage{multicol}

\usepackage[stable]{footmisc} % footnotes in section tags

\setlength{\parindent}{0pt}

%%% Headers, footers
\author{Niclas Olofsson}
\pagestyle{fancy}
\renewcommand{\headrulewidth}{1pt}
\fancyhead[LO,L]{Automated testing of a dynamic web application}
\lfoot{}\cfoot{\thepage}\rfoot{}

%%% Custom chapter and section style
\usepackage{titlesec, blindtext, color}
\definecolor{gray75}{gray}{0.75}
\newcommand{\hsp}{\hspace{20pt}}

\usepackage{sectsty}
\allsectionsfont{\sffamily\mdseries\upshape} % sans serif headings

\newcommand{\quotes}[1]{``#1''}

\titleformat{\chapter}[hang]{\Huge\bfseries}{\thechapter\hsp\textcolor{gray75}{|}\hsp}{0pt}{\Huge\bfseries}

\lstset{
language=Ruby,
basicstyle=\small\sffamily,
numbers=left,
numberstyle=\tiny,
frame=tb,
columns=fullflexible,
showstringspaces=false
}

%%% Title page
\makeatletter
\newcommand{\myinfo}{
  \gdef\@myinfo{%
  \begin{tabular}{cl}
  Technical  & Mattias Ekberg\\
  supervisor & GOLI AB\\
  \\
  Supervisor & Anders Fröberg\\
             & IDA, Linköping University\\
  \\
  Examiner & Erik Berglund\\
           & IDA, Linköping University\\
  \end{tabular}
  \par\vspace{12pt}
  \textsc{Linköping University\\
          Department of Computer and Information Science}}
}
% add the extra information after the date
\postdate{\par\vspace{12pt}\@myinfo\end{center}}
\makeatother

\title{\textsc{Master's thesis}\\\Huge\textbf{Automated testing of a dynamic web application}}
\author{Niclas Olofsson}
\date{\today}

\myinfo

\begin{document}

\renewcommand{\lstlistingname}{Code listing}

\maketitle
\newpage

\begin{abstract}
Software testing plays an important role in the process of verifying
software functionality and preventing bugs in production code. By
writing automated tests using code instead of conducting manual tests,
the amount of tedious work during the development process can be reduced
and the software quality can be improved.\\

This thesis presents the results of a conducted case study on how
automated testing can be used when implementing new functionality in a
Ruby on Rails web application. Different frameworks for automated
software testing is used as well as test-driven development methodology,
with the purpose of getting a broad perspective on the subject. We study
common issues with testing in these kinds of applications, and discuss
drawbacks and advantages of different testing approaches. We also look
into quality factors which are applicable for tests, and analyze how
these can be measured.\\

\end{abstract}

% Fake that the Acknowledgments is an abstract in order to get a nice style
\renewcommand{\abstractname}{Acknowledgments}

\begin{abstract}
This final thesis was conducted at GOLI AB, on the business incubator
LEAD during the spring of 2014. I would like to thank my dear colleagues
Lina, Malin and Madeleine, as well as all other people at LEAD for good
fellowship and encouragement during this period. I specially want to
thank my technical supervisor Mattias for help, support, interesting
discussions and ideas.\\

I would also like to give thanks to my supervisor Anders Fröberg as well
as my examiner Erik Berglund for overall help and support.\\

\end{abstract}

\setcounter{tocdepth}{4}
\tableofcontents
\thispagestyle{empty} % No page numbering of ToC pages
\newpage

\setcounter{page}{1}

\chapter{Introduction}

  \section{Background}
  \label{sec:background}
  During code refactoring or implementation of new features in software,
errors often occur in existing parts. This may have a serious impact on
the reliability of the system, thus jeopardizing user's confidence for
the system. Automatic testing is utilized to verify the functionality of
software in order to detect bugs and errors before they end up in a
production environment.\\

The commissioner body of this project, GOLI, is a startup company
developing a web application for production planning. Starting new web
application companies often means rapid product development in order to
create the product itself, while maintenance levels are low and the
quality of the application is still easy to assure by manual testing. As
the application and the number of users grows, maintenance and bug
fixing becomes an increasing part of the development. The size of the
application might make it implausible to test in a satisfying way by
manual testing.\\

Due to requirements from customers, GOLI wishes to extend the code base
of the web application to include new features for handling staff
manning. The current system uses automatic testing to some extent, but
these tests are cumbersome to write and takes long time to run. The
purpose of the thesis is to analyze how this application can begin using
tests in a good way whilst the application is still quite small. The
goal is to determine a solid way of implementing new features and bug
fixes in order for the product to be able to grow effortlessly.\\


  \section{Problem formulation}
  
The goal of this final thesis is to analyze how automated tests can be
introduced in an existing web application, in order to detect software
bugs and errors before they end up in a production environment. In order
to do this we will conduct a case study focused on how this can be done
in the GOLI production planning system. We use the results from this
case study in order to discuss how testing can be applied to dynamic
web applications in general.\\

The main research question is thus to determine how testing can be
introduced in the scope of the GOLI ECP application. The focus should
be on techniques which are relevant this specific application, i.e.
testing web applications that uses Ruby on Rails
\footnote{Ruby on Rails framework, \url{http://rubyonrails.org/}}
and KnockoutJS
\footnote{KnockoutJS framework, \url{http://knockoutjs.com/}} with a
MongoDB\footnote{MongoDB NoSQL database, \url{https://www.mongodb.org/}}
database system for data storage. We will investigate which kind of
problems that are specific to this kind of environment, and attempt to
find suitable solutions for them.\\


  \section{Scope and limitations}
  There exists different categories of software testing, for example
performance testing and security testing. The scope of this thesis is
regression- or quality assurance testing\footnotemark, in which the
purpose is to verify the functionality of a part of the system rather
than measuring its characteristics. We will also only cover automatic
testing, as opposed to manual testing where the execution and result
evaluation of the test is done by a human. The term \emph{testing} will
hereby refer to automatic software regression testing unless specified
otherwise.\\

Since the result of this thesis will be evaluated in specific web
application (i.e. GOLI Kapacitetsplanering), we will only cover
techniques which are relevant this specific application. In other words,
we will focus on testing web applications which uses Ruby on
Rails\footnotemark and KnockoutJS\footnotemark.\\

\footnotetext{This is sometimes also called functional testing, but we
will refrain from using that term since it has different meanings to
different people.}

\footnotetext{Ruby on Rails framework, \url{http://rubyonrails.org/}}

\footnotetext{KnockoutJS framework, \url{http://knockoutjs.com/}}



  \section{Conventions and intended audience}
  
The intended audience of this report is primarily people with some or
good knowledge of programming and software development. A great deal of
knowledge in the area of software testing or test methodologies should
however not be required. The report can probably be of interest for
people without programming knowledge which is interested in the area of
software testing and development.\\

Code examples are written in Ruby\footnote{\url{https://www.ruby-
lang.org}} since that is the primary language of this thesis, but with
emphasis on being understandable by people without knowledge of this
language rather than using Ruby-specific tools and practices.\\

The area of software development contains several terminologies which
are similar or exactly the same. In cases where multiple different
terminologies exists for a certain concept, we have chosen the term with
most hits on Google. The purpose of this was to choose the most widely-
used term, and the number of search results seemed like a good measure
for this. Footnotes with alternative terminologies is present where
applicable.\\



\chapter{Methodology}
  
This chapter outlines the general research methodology of this thesis,
and explains different choices. The methodology of this thesis is
generally based on the guidelines proposed by
\citet{article:casestudies} for conducting a case study. An objective is
defined and a literature study is conducted in order to establish a
theoretical base. A case study is then conducted in order to evaluate
the theory by applying it in a real-life application context. Finally,
the result is analyzed in order to draw conclusions about the theory.


  \section{Literature study}
  
The literature study was based on the problem formulation, and therefore
focuses on web application testing overall and how it can be automated.
In order to get a diverse and comprehensive view on these topics,
multiple different kinds of sources were consulted. As a
complement to a traditional literature study of peer-reviews articles
and books, we chosen to also consider blogs and video recordings of
conference talks.\\

While blogs are not either published nor peer-reviewed, they often
express  interesting thoughts and ideas, and may often give readers a
chance to leave comments and discuss its contents. This might not
qualify as a review for a scientific publication, but it also gives
greater possibilities of leaving feedback on outdated information and
is more open for discussion than traditional articles. Conference
talks has similar properties.\\

Blogs and conference talks does have another benefit over articles and
books since they can be published instantly. The review- and publication
process for articles is long and may take several months, and also might
not be available in online databases until after their embargo period
has passed \cite{wiki:embargo,pdf:publishing}. This makes it hard to
present relevant and up-to-date on topics such as web development, where
the most recent releases of well-used frameworks are less than a year
old \cite{wiki:rails-versions,wiki:django-versions,web:knockout-versions}.\\

Sources are mainly relied upon recognized people in the open-source
software community. Due to this, one might notice a skew towards to
agile approaches and other often ideas often used in this community.\\


  \section{Choices}
  
Selection of the technologies to use in a certain project is one of the
most important steps, since it affects the rest of the development
process. In this case, the commissioning body of this thesis mainly gave
the choice of frameworks for development since the existing software
used certain programming languages and frameworks. The main server-side
was written in Ruby using the Ruby on Rails framework, and the client-
side code was written in CoffeeScript\footnote{CoffeeScript is a
scripting language that compiles into Javascript.} using the
Knockout.js framework.\\

For the choice of testing-related frameworks, we chose to look for
frequently used and active developed open source frameworks.
Technologies that are used by many people intuitively often have more
resources on how they are used, and also have the advantage of being
more likely to be recognized by future developers.\\

Active development is another crucial property of used frameworks.
Unless a framework is updated continuously, it is likely to soon be
incompatible with future versions of other frameworks, such as Rails.
Another benefit is that new features and bug fixes are released.\\

The Ruby Toolbox website\footnote{\url{https://www .ruby-toolbox.com/}},
which uses information from the Github and RubyGems websites, was
consulted in order to find frameworks with mentioned qualities.\\



\chapter{Theory}
\label{chap:theory}

  \section{Levels of testing}
    
\label{sec:theory_levels}

One fundamental part of all software development is the concept of
abstraction. Abstraction can be described as a way of decomposing an
application into different levels, with different level of detail. This
permits the developer to ignore certain details of the software, and
instead focus on other details.\\

Consider the development of a simple game with basic graphics. On the
lowest level possible, a such game requires a tremendous amount of work
in order to shuffle data between hardware buses, perform memory accesses
and CPU operations. By using higher abstraction levels, one can use
third-party frameworks for drawing graphics to the screen and detecting
collisions. The operating system and programming language takes care of
handling bus-accesses and memory management. This allows the developer
to focus on designing the game logic itself, rather than bothering with
drawing individual pixels or figuring out where in the memory to store
data. \cite{paper:abstraction}\\

In the same way, testing can be performed at several different levels.
There are several ways of defining these levels, but one way of
describing it is like a pyramid as seen in \fref{fig:testing_pyramid}.
We can imagine testing at different levels as holding a flashlight at
different levels of the pyramid. If we hold the flashlight at the top of
the pyramid, the flashlight will illuminate a large part of the pyramid.
If the flashlight is hold at the bottom of the pyramid, a much smaller
piece of the pyramid will be illuminated. Similar to this, testing at a
high level permit us to ignore a lot of details. Due to the high level
of abstraction, a large part of the code must be run in order for the
test to be completed. Testing at a lower level requires a much smaller
part of the code to be run. Different levels of testing have different
advantages, drawbacks and uses, which are covered in the following
subsections.\\

\begin{figure}
\centering
\includegraphics[width=0.7\textwidth]{theory/levels/triangle}
\caption{The software testing pyramid, with two flashlights at different
         levels illustrating how the level of testing affects the amount
         of tested code.}
\label{fig:testing_pyramid}
\end{figure}


    \subsection{Unit-testing}
    \label{sec:unit_testing}
    \MakeShortVerb{\|}

% In order to verify the functionality of program, it is possible to use a
% formal logical proof. This is done dividing the program into small
% pieces of logic, and the possible input data into different classes.
% \citet{book:pfleeger}

Unit testing\footnote{Also called low-level testing, module testing or
component testing} refers to testing of small parts of a software. In
practice, this often means testing a specific class or method of a
software. The purpose of unit tests is to verify the implementation,
i.e. to make sure that the tested unit works correct.
\cite{web:xp_unittests, book:pfleeger}.\\

Since each unit test only covers a small part of the software, it is
easy to find the cause for a failing test. Writing unit tests alone does
not give a sufficient test coverage for the whole system, since unit
tests only assures that each single tested module works as expected. A
well- tested function for validating a 12-digit personal identification
number is worth nothing if the module that uses it passes a 10-digit
number as input \cite{wiki:unittests}.\\


\subsection{Testability}

Writing unit tests might be really hard or really easy depending on the
implementation of the tested unit. \citet{video:misko_psychology} claims
that it is impossible to apply tests as some kind of magic after the
implementation is done. He demonstrates this by showing an example of
code written without tests in mind, and points out which parts of the
implementation that makes it hard to unit-test.
\citeauthor{video:misko_psychology} mentions global state variables,
violations of the Law of Demeter, global time references
and hard-coded dependencies as some causes for making implementations
hard to test.\\

Global states infers a requirement on the order the tests
must be run in, which is bad since it is often non-deterministic and
therefore might change between test runs. Global time references is bad
since it depends on the time when the tests are run, which means that a
test might pass if it is run today, but fail if it is run tomorrow.\\

The Law of Demeter\footnote{Also called the principle of least
knowledge} means that one unit should only have limited knowledge of
other units, and only communicate with related modules
\cite{wiki:demeter}. If this principle is not followed, it is hard to
create an isolated test for a unit which does not depend on unrelated
changes in some other module. The same thing also applies to the usage
of hard-coded dependencies. This makes the unit dependent on other
modules, and makes it impossible to replace the other unit in order to
make it easier to test.\\

\citeauthor{video:misko_psychology} shows how code with these issues can
be solved by using dependency injection and method parameters instead
of global states and hard-coded dependencies. This makes testing of the
unit much easier.\\


\subsection{Stubs, mocks and fakes}

Another way of dealing with dependencies on other modules is to use some
kind of object that replaces the other module. The replacement object
has a known value that is specified in the test, which means that
changes to the real object will not affect the test. The reasons for
using replacement objects is often to make it more robust to code
changes outside the tested unit. It might also be used instead of calls
to external services such as web-based API:s in order to make the tests
run when the service is unavailable, or to be able to test
implementations that depends on classes that has not been yet.\\

The naming of different kinds of replacement objects may differ, but two
often used concepts are \emph{stubs} and \emph{mocks}. Both these
replacement objects are used by the tested unit instead of some other
module, but mocks sets expectations on how it can be used by the tested
module on beforehand. \cite{web:mocks_arent_stubs}\\

Another type of replacement object are \emph{fakes} or \emph{factory
objects}. This kind of object is typically provides a real
implementation of the object that it replaces, as opposed to a stub or
mock which only has just as many methods or properties that is needed
for the test to run. The difference between a fake object and the real
object is typically that fake objects uses some shortcut which does not
work in production. One example is objects that are stored in memory
instead of in a real database, in order to gain performance.
\cite{web:mocks_arent_stubs}\\

\citet{video:boundaries} mentions some of the drawbacks with using mocks
and stubs. If the interface of the replaced unit changes, this might not
be noticed in our test. Consider the scenario given in \ref{lst:mocks}.
In this example we have written a test for the |take_cookie()| method of
the |CookieJar| class, which replaces the |eat_cookie()| method with a
stub in order to make the |CookieJar| class independent of the |Cookie|
class.\\

If we rename the |eat_cookie()| method to |eat()| without changing the
test or the implementation of |take_cookie()|, the test might still pass
although the code will fail in a production environment. This is since
we have mocked an object which no longer exists in the |Cookie| class.\\

Some testing frameworks and plug-ins detects replacement of non-existing
methods and warns or makes the test fail if such mocks are created
\cite{video:boundaries}. Another possible solution is to re-factor the
code to avoid the need for mocks or stubs.\\

\begin{lstlisting}[caption=Example of how mocking might make tests pass
                           even when they are not indented to,
                   label=lst:mocks, float=t]
class Cookie
    def eat_cookie
        "Cookie is being eaten"
    end
end

class CookieJar
    def take_cookie
        self.cookies.first.eat_cookie
        "One cookie in the jar was eaten"
    end
end

def test_cookie_jar
    Cookie.eat_cookie = Mock
    assert CookieJar.new.take_cookie == "One cookie in the jar was eaten"
end
\end{lstlisting}


    \subsection{Integration-testing}
    \label{sec:integration_testing}
    Since a unit test only assures that a single unit works as expected,
faults may still reside in how the units works together. The purpose of
integration testing is to test several individual units together, in
order to see if they still work together as intended.\\

There are several ways of performing integration testing, as well as
arguments and opinions about the different ways. \citet{book:adp} state
that integration tests should be built incrementally by extending unit
tests. The unit tests are extended and merged so that they span over
multiple units. The scope of the tests is increased gradually, and both
valid and invalid data is given into the integration tested system unit
in order to test the interfaces between smaller units. Since this
process is done gradually, it is possible to see which parts of the
integrated unit that is faulty by examining which parts have been
integrated since the latest test run.\\

\citet{book:pfleeger} refer to the type of integration testing described
by \citeauthor{book:adp} as \emph{bottom-up testing}, since several
low-level modules (modules at the bottom level of the software) are
integrated into higher-level modules. Multiple other integration
approaches such as \emph{top-down testing} and \emph{sandwich testing}
are also mentioned, and the difference between the approaches is in which
order the units are integrated.\\

\citet{video:integrated_scam} criticizes one kind of integration tests,
which he refers to as \emph{integrated tests}. This refers to
integration tests that are testing the functionality of multiple units
in the same way as unit tests, i.e. by input data and examine the
output. When testing multiple units in this way, one loses the ability
to see which part of all the tested units that are actually failing. As
the number of tested units rises, the number of possible paths will grow
exponentially. This makes it hard to see the reason for a failed test,
but also makes it very hard to decide which of all this paths that needs
to be tested. \citeauthor{video:integrated_scam} claims that this fact
makes developers sloppier, which increases the risk of introducing
mistakes that goes unnoticed through the test suite. If the problem is
solved by writing even more integration tests, developers have less time
to do proper unit tests and instead introduce more sloppy designs. One
may however argue that this argument is based on the fact that
integrated tests to a large part are used instead of unit tests, rather
than as a complement as suggested by \citeauthor{book:pfleeger} and
\citeauthor{book:adp}.\\

Instead of integrated tests, another type of integration tests called
contract- and collaboration tests are proposed. The purpose of these
tests is to verify the interface between all unit-tested modules by
using mocks to test that Unit A tries to invoke the expected methods on
Unit B (contract test). In order to avoid errors due to mocking, tests
are also needed to make sure that Unit B really responds to the calls
that are expected to be performed by Unit A in the contract test. The
idea of this is to build a chain of trust inside our own software via
transitivity. This means that if Unit A and Unit B works together as
expected, and Unit B and Unit C also works together as expected, Unit A
and Unit C will also work together as expected.\\


    \subsection{System-testing}
    System testing is conducted on the whole, integrated software system.
Its purpose is to test if the end product fulfills specified
requirements, which includes determining whether all software units (and
hardware units, if any) are properly integrated with each other. In some
situations, parameters such as reliability, security and usability is
tested.\cite{book:adp}\\

The most relevant part of system testing for the scope of this thesis is
functional testing. The purpose of this is to verify the functional
requirements of the application at the highest level of abstraction. In
other words, one wants to make sure that the functionality used by the
end-users works as expected. This might be the first time where all
system components are tested together, and also the first time the
system is tested on multiple platforms. Because of this, some types of
software defects may never show up until system testing is
performed.\cite{book:adp}\\

\citeauthor{book:adp} proposes that system testing should be performed
as black box tests corresponding to different application use-cases. An
example could be testing the functionality of an online library catalog
by adding a new user to the system, log in and perform different types
of searches in the catalog. Domain-testing techniques such as boundary-
value testing are used in order to narrow down the number of test
cases.\\



  \section{Software development methodologies}
    
During the ages of computers and software development, several software
development methodologies have been proposed. A software development
methodology defines different activities and models that can be
followed during software development. Such activities can for instance
be defining software requirements or writing software implementations,
and the methodology typically defines a process with a plan for how and
in which order the activities should be done. The waterfall model and
the V model are two classical examples of software development
methodologies. \cite{article:sw_methodologies}\\

Extreme programming (XP) is a software development methodology created
by Kent Beck, who published a book on the subject in 1999
\cite{wiki:xp}. This methodology was probably the first to propose
testing as a central part of the development process, as opposed to
being seen as a separate and distinct stage
\cite{article:sw_methodologies}. These ideas was later further
developed, and the concept of \emph{testing methodologies} was founded.
A testing methodology defines how testing should be used within the
scope of a software development methodology, and the following sections
will focus on the two most prominent ones.\\


    \subsection{Test-driven development}
    \label{sec:tdd}
    
We have striven for using test- and behavioral-driven development
methodologies in as strict way as possible throughout this entire
thesis. The purpose of this has been to gain experience of different
advantages and drawbacks of these methodologies. In situations where we
have found these methodologies to be unsuitable, we have however chosen
to disregard some principles of these methodologies rather than try to
apply them anyway.\\

Using a test-driven approach has is in our opinion two major advantages.
The most important is perhaps that it mitigates a sense of fear about
whether or not the implementation really works as it is supposed to, and
a fear of that things may break when code is refactored. The other
important aspect is that the written code is self-testing; you can run a
command in order to find out whether or not the code is in a working
state, rather than just test manually and see what happens.\\

Another benefit from using the test-first principle is that one often
think through the design before implementation. We have to write the
implementation in a way such that it is testable, which makes us
discover new ways of solving the problem and design our software better.
We also usually find it tedious to write tests when the implementation
is done, and we think that writing the tests first simply makes the
development more fun than writing tests after implementation.\\

It is however no axiomatic guarantee for that a more testable design is
better than a less testable design. For example, making it possible to
test a module may require it to be split up in several non-intuitive
modules, or to use a large amount of mocking. We believe that it is better
to test on a higher level and perhaps discard the test-first principle
in cases where making the code testable introduces more drawbacks than
advantages to the overall design.\\


    \subsection{Behavior-driven development}
    Behavior-driven development (BDD) is claimed to originate from an
article written by \citet{web:dan_north}, and is based on
TDD \cite{wiki:bdd}. This section is based upon the original article
written by \citeauthor{web:dan_north}.\\

\citeauthor{web:dan_north} describes that several confusions and
misunderstandings often appeared when he taught TDD and other agile
practices in projects. Programmers had trouble to understand what to
test and what not to test, how much to test at the same time, naming
their tests and understanding why their tests failed.
\citeauthor{web:dan_north} thought that it must be possible to introduce
TDD in a way that avoids these confusions.\\

Instead of focusing on what test cases to write for a specific feature,
BDD instead focuses on behaviors that the feature should imply. Each
test is described by a sentence, typically starting with the word
\emph{should}. For a function calculating the number of days left until
a given date, this could for example be \quotes{should return 1 if
tomorrow is given} or \quotes{should raise an exception if the date is
in the past}.\\

Many frameworks use strings for declaring the sentence describing each
test. Using strings rather than traditional function names allows us to
use a native language, and solves the problem of naming tests. The
string describing the behavior is used instead of a traditional function
name, which also makes it possible to give a human-readable error if the
module fails to fulfill some behavior. This can make it easier to
understand why a test fails. It also sets a natural barrier for how
large the test should be, since it must be possible to describe in one
sentence.\\

After coming up with these ideas, \citeauthor{web:dan_north} met a
software analyst and realized that writing behavior-oriented
descriptions about a system had much in common with software analysis.
Software analysts write specifications and acceptance criteria for
systems before development, which are used when developing the system as
well as for evaluating the system during acceptance testing. They came
up with a way of writing scenarios in order to represent the purpose and
preconditions for behaviors in a uniform way, as seen below.\\

\framebox[\textwidth]{
\minibox{
    \emph{Given} some initial context (the givens),\\
    \emph{When} an event occurs,\\
    \emph{Then} ensure some outcomes.\\
}}
\linebreak
\linebreak

By using this pattern, analysts, developers and testers can all use the
same language, and is hence called a \emph{ubiquitous language} (a
language that exists everywhere). Multiple scenarios are written by
analysts to specify the properties of the system, which can be used by
developers as functional requirements, and as desired behaviors when
writing tests.\\



  \section{Evaluating test efficiency}
    \label{sec:efficiency}
    
\subsection{Usability of test coverage}

As seen in \fref{sec:results_coverage}, the test coverage for the
existing RSpec integration-tests was 42 \% before this thesis project,
which is quite much considering that the existing tests was partially
duplicated and only was focused on a few specific parts of the system.
One reason for this is found when the coverage reports are inspected,
where we can see that certain kinds code gets full statement coverage,
even though there are not any tests for them. Method definitions and
field definitions on model objects are such types, since they are
executed as soon the application loads.\\

A major part of the modules with high test coverage score in the
beginning of the case study was modules without any methods, such as
data models. Since these modules also can have methods which needs
tests, it is not possible to just exclude all such modules from the
coverage report either. This is an symptom of the weakness of statement
coverage, which is also previously discussed in
\fref{sec:theory_statement_coverage}. Statement coverage is a very weak
metric and it is absolutely possible to create a non-empty class with
100 \% statement coverage without writing any tests for it.\\

One may thus impeach the usefulness of statement test coverage. We did
however find that apart from the coverage score being a bad indicator on
how well-tested the code was, analyzing the statement coverage report
was highly usable for finding untested parts of the code. The nature of
our code was such that it in general contained few branches, which of
course makes statement coverage considerably much more useful than for
code with higher complexity.\\

On the client-side, we also had the ability to evaluate branch coverage.
The most important difference between the coverage scores for branch
coverage versus statement coverage is that the branch coverage is zero
for all untested functions, which makes the branch coverage score a more
sound metric for the overall testing level. Branch coverage of course
also made it possible to discover a few more untested code paths.\\


\subsection{Usability of mutation testing}

We believe that mutation testing may be a good alternative to code
coverage as method for evaluating test efficiency. Our small manual test
indicated that mutation testing and test coverage is related, since
several of the alive mutants modified the same line, which also showed
to have zero statement test coverage. The mutation tests also found non-
equivalent modifications to the code which was not discovered by the
tests, even though they had full statement and branch test coverage.\\

Our experiences with mutation testing however shows that more work is
required before it is possible to use it in our application. Mutation
tests may also have limited usability on the server side, which in our
case is mostly tested with higher-level integration-tests rather than
with unit-tests. As mentioned in \fref{sec:theory_mutation}, efficient
mutation testing requires the scope to be narrow and the test suite to
be fast, due to the large amount of possible mutants. This is not the
case for integration-level tests. Mutation testing might thus be more
useful for the client-side, which contains more logic and therefore is
tested with isolated unit-tests to a larger extent.\\


\subsection{Efficiency of tests written during this thesis}

Since most of the implemented code consisted of extensions or
refactoring to existing files, it is hard to get a fair metric value for
how well tested the modified parts of the software is. This is because
these metrics are given in percent per file, and since the same file
contains a lot of unmodified code, that will affect the result. It is
also hard to detect which parts of the code that has been modified. We
ended up using a self-constructed, subjective metric, which imposes
questions on whether or not we can draw any solutions about how
efficient the tests for the new functionality is.\\

We can however see that the overall test coverage has increased




  \section{Other test quality factors}
    \label{sec:quality}
    \subsection{Test coverage}

Test coverage\footnote{Also known as \emph{code coverage}} is a measure to
describe to which degree the program source code is tested by a certain
test suite.\cite{wiki:coverage}

However, \citeauthor{book:art_of_testing} shows that a simple
20-statement program consisting of a single loop with a couple of nested
if-statements can have 100 trillion different logical paths. While real-
world programs might not have such extreme amounts of logical paths,
they are typically much larger and more complex. In other words can it
be close to impossible to achieve full path test coverage in many
cases.\\





\chapter{Approach}

  Based on the theory discussed in \ref{chap:theory}, this chapter
outlines a hypothesis about how testing of the ECP system should be
implemented. We also constitute activities that should be done during
the case study, and thus forms the base for the results presented in
\ref{chap:results}. In \ref{chap:discussion}, we compare the outlined
hypothesis described in this chapter to the actual results in order to
see how well this hypothesis worked in practice.


  \section{Hypothesis}
  \label{sec:hypothesis}
  
\begin{itemize}
    \item A test-driven development methodology is used during
          development, and tests are written before implementation.
    \item The major parts of all written tests are low-level unit tests.
    \item A few system level tests are written in order to test the
          integration of all units together.
    \item System level tests are run with Selenium in order for them
          to detect cross-browser system bugs.
    \item Test coverage should be used for measuring test quality.
\end{itemize}


  \section{Case study}
  
The case study is divided into three sub parts. The purpose of each
sub part is to evaluate some aspects of the testing approach, in order
to compose a good evaluation of the chosen testing approach when
combined.\\


\subsection{Refactoring of old tests}

There have been attempts to introduce testing of the ECP system a while
back, but this has stopped since the chosen approaches was found to be
very cumbersome. At the start of this project, the implemented tests had
not been maintained for a very long time, which resulted in that many
tests failed although the system itself worked fine.\\

As mentioned in section \ref{sec:hypothesis}, TDD methodology is
used during the case study. This methodology is based on the fact that
tests are written before implementation of new features and then run
iteratively during development. The test suite should pass, then fail
after a new test has been implemented, and then pass again after the new
feature has been implemented. This of course presupposes that existing
tests can be run and give predictable results.\\

The first part of the case study is therefore to make all old tests run.
Apart from being a prerequisite for new tests and features to be
implemented, it also gives a view on how tests are affected as new
functionality is implemented. This is especially interesting since it
otherwise would be impossible to evaluate such factors in the scope of a
master's thesis. It also gives a perspective on some of the advantages
and drawbacks of the old testing approach.\\

Another drawback of the old tests are the fact that they run too slow in
order to be continuously in a TDD manner. Another objective of this part
of the case study is therefore to make them faster, so at least some of
the tests can be run continuously.\\


\subsection{Implementation of new functionality}

As mentioned in section \ref{sec:background}, the commissioner body of
this project wishes to implement support for staff manning in the ECP
system. This functionality is implemented and tests are written for
new parts of the system as well as for re-factored code.\\

The purpose of this part is, besides implementing the new feature
itself, to evaluate test-driven development and how tests and
implementation code can be written together by using TDD methodology and
an iterative development process. We also gain more experience of
writing unit tests in order to evaluate how different kinds of tests
serves different purposes in the development process.\\


\subsection{Increasing test coverage}

In order to evaluate the tests written in previous parts of the case
study, code coverage is used as a measure. The last part of the case
study focuses on analyzing parts of the application that is untested or
very weakly tested, and write tests for them. We also evaluate the tests
written in previous parts of the case study and complement them if
needed.\\

The purpose of this part is to get a more solid experience of using test
coverage as a measure for code quality, and to produce a measurable
output of the case study. We also look at how TDD methodology works in
an situation where the functionality is already implemented without
tests, as opposed to using it when implementing new functionality.\\



\chapter{Results}
\label{chap:results}

  This chapter presents the results of the case study. (wip)


  \section{Tools and frameworks}

  \section{Chosen test approaches}

  \section{Test efficiency}
  
There are several metrics for evaluating different quality factors of
written tests. One key property of tests is that they must be able to
detect defects in the code, since that is typically the very main reason
for writing tests at all. We have chosen to call this property
\emph{test efficiency}. Another quality factor is the performance of the
tests, i.e. the execution time of the test suite. This section explains
the properties of these quality factors, as well as their purpose.\\

Readability and ease of writing of tests is another important property
which are not mentioned in this section, since it is hard to measure in
an objective way. It also often depends on the testing framework rather
than on the tests themselves. We evaluate these properties for each
chosen testing framework instead.\\

\subsection{Test coverage}
\label{sec:coverage}
\label{sec:results_coverage}

A quick overview of the results are presented in
\fref{tab:unit_coverage}. More information is presented in the following
sections. As discussed in \fref{sec:coverage_frameworks}, we were unable
to find any tool for analyzing anything else than statement test
coverage for Ruby.\\

\begin{table}[t]
    \centering
    \begin{tabular}{l l l}
        Phase & No. of tests & Test coverage\\
        \hline
        Before case-study &       59 & 42.24 \%\\
        After first part  &       58 & 55.25 \%\\
        After second part &       81 & 62.34 \%\\
    \end{tabular}
    \caption{ Statement test coverage of RSpec unit- and integration tests at different phases. }
    \label{tab:unit_coverage}
\end{table}

\begin{table}[t]
    \centering
    \begin{tabular}{l l l}
        Description & No. of tests & Test coverage\\
        \hline
        RSpec browser tests &     3 & 46.3 \%\\
        Cucumber browser tests &  8 & 61.2 \%\\
        All RSpec tests &        84 & 66.8 \%\\
    \end{tabular}
    \caption{ Statement test coverage including browser tests after the second part of the case study. }
    \label{tab:browser_coverage}
\end{table}

\subsubsection{Before the case study}

Before the start of this master's thesis, the average statement coverage
of all RSpec unit- and integration tests was 42 \%.\\

For the client-side code, no unit-tests existed. The test coverage was
thus zero at this state.\\


\subsubsection{After the first part of the case study}

The first part of the case study (described in \fref{sec:casestudy_1})
was focused on rewriting broken tests, since some of the existing tests
was not functional. During this period, a large part of the existing
test suite was either fixed or completely rewritten. Many of the
existing unit- and integration tests was rewritten and a few large
acceptance-level test was replaced by more fine-grained integration
tests. The statement coverage of the increased to 53 \% after this
part of the case study.\\

There were still no client-side unit-tests after this part, since the
focus was fixing the existing server-side tests. The client-side test
coverage was still zero at this state \\

\subsubsection{After the second part of the case study}
\label{sec:result_coverage_end}

The second part of the case study (described in \fref{sec:casestudy_2})
was focused on implementing new functionality while re-factoring old
parts as needed and write tests for new as well as re-factored code. The
first sprint of this part was focused on basic functionality, while the
second sprint was focused on extending and generalizing the new
functionality.\\

When the implementation of the new functionality was finished, statement
coverage of unit- and integration tests for the server side was 62 \%. A
subjective measure indicated that new and re-factored functions in
general has high statement coverage. In cases where full statement
coverage is not achieved, the reason is generally special cases. The
most common example is when error-messages are given when request
parameters are missing or invalid.\\

Statement test coverage for the client-side was 21 \% and the
corresponding branch coverage was 10 \%. A subjective measure
indicates that almost all of the newly implemented functions achieves
full statement coverage. The branch coverage is in general also high for
newly implemented functions, but conditionals for special cases (such as
when variables are zero or not set) are sometimes not covered.\\

\subsubsection{Test coverage for browser tests}

All metrics in the previous sections refer to test coverage for unit-
and integration tests. As one of the last steps of the second part of
the case study, browser tests was added in order to do system-level
testing.\\

The total statement test coverage for the RSpec browser tests alone was
46 \%, and the statement test coverage for all Ruby tests was 67 \%.
With the tools used, it was not possible to measure Javascript test
coverage for the test suite including browser tests.\\

The total statement test coverage for the Cucumber browser tests alone
was 61 \%. It was not possible to calculate the total coverage for
Cucumber and RSPec tests combined.\\


\subsection{Mutation testing}
\label{sec:theory_mutation}
An alternative to draw conclusions from which paths of the code that is
run by a test, as done when using test coverage, is to draw conclusions
from what happens when we modify the code. The idea is that if the code
if incorrect, the test should fail. Thus, we can modify the code so it
becomes incorrect and then look at whether the test fails or not.\\

Mutation testing is done by creating several versions of the tested
code, and introduce slight modifications into each one of them. Each
such version containing a mutated version of the original source code is
called a \emph{mutant}. All tests which we want to evaluate are run for
each mutant. If at least one of the tests fails, the mutant is
considered to be \emph{killed}. We can measure the efficiency of the
test suite as the ratio of killed mutants versus the total number of
mutants.\cite{wiki:mutation}\\


\begin{lstlisting}[caption=Example of a piece of code before mutation,
                   label=lst:mutation_before, float=t]
    def odd?(x, y)
        (x % 2) && (y % 2)
    end
\end{lstlisting}


\begin{lstlisting}[caption=Two mutated versions of \ref{lst:mutation_before},
                   label=lst:mutant_1, float=t]
    def odd?(x, y)
        (x % 2) && (x % 2)
    end

    def odd?(x, y)
        (x % 2) || (y % 2)
    end
\end{lstlisting}


\subsection{Execution time}

All execution times below are mean values of multiple runs and excluding
start-up time. Execution times was measured on a 13" mid-2012 Macbook
Air\footnote{\url{http://www.everymac.com/systems/apple/macbook-
air/specs /macbook-air-core-i5-1.8-13-mid-2012-specs.html}} with 1.83
GHz Intel Core i5 processor and Mac OS 10.9.2. Google Chrome 34.0.1847
was used as browser for execution of browser tests and Jasmine unit
tests. The browser window was focused in order to mitigate effects of
Mac OS X power saving features.\\

\subsubsection{Before the case study}

The total running time of the 59 RSpec integration- and unit tests
before the case study was 23.0 seconds. Average test execution time was
340 ms per test, which means an execution rate of 3.0 tests per
second.\\

Twelve Cucumber browser tests existed at this time (with 178 steps), but
none of these tests passed at this stage. After fixing the most critical
issues with these tests, 36 steps passed in 42.7 seconds. The average
time per test was 3.6 seconds. These numbers are however not entirely
fair compared to later test runs, since large parts of the tests could
not be run at this stage.\\


\subsubsection{After refactoring old tests}

Several of the previous tests was removed and replaced by more efficient
tests during the second part of the case study. 58 RSpec integration-
and unit tests existed after this stage, an the total execution time was
8.3 seconds.\\

The running time of the ten Cucumber tests (with 118 steps) was 160
seconds, and the average time per test was 18 seconds.\\


\subsubsection{After implementation of new functionality}

In total, 81 RSpec integration- and unit tests existed at this stage,
and the execution time of these was 7.6 seconds. Average test execution
time was 94 ms per test, and the rate was 10.7 tests per
second.\\

At this phase, 34 Jasmine unit tests had been written and the total
running time of these was 0.043 seconds. Average execution time was 1.3
ms per test and the rate was 791 tests per second.\\

The running time of the three RSpec browser tests was 12.4 seconds, with
an average time per test of 4.1 seconds per test. One of the Cucumber
tests had been re-factored into RSpec browser tests, and the running
time of the remaining nine Cucumber browser tests (with 111 steps) was
153 seconds. The average time per test was 17 seconds.\\


\begin{table}[t]
    \centering
    \begin{tabular}{l l l l l}
        Phase & No. of tests & Total time (s) & Time per test (ms) & Tests per second\\
        \hline
        Before case-study & 59 & 19.9& 340 & 3.0 \\
        After first part  & 58 & 8.3 & 140 & 6.9 \\
        After second part & 81 & 7.6  & 153 & 10.7\\
    \end{tabular}
    \caption{ Execution times of RSpec integration- and unit tests at different phases }
    \label{tab:unit_times}
\end{table}


\begin{table}[t]
    \centering
    \begin{tabular}{l l l l l}
        Phase & No. of tests & Total time (s) & Time per test (ms) & Tests per second \\
        \hline
        Before case-study & 12 & 43 & 3.6 & 0.27 \\
        After first part  & 10 & 160 & 18 & 0.056\\
        After second part & 9 &  153 & 17 & 0.056\\
    \end{tabular}
    \caption{ Execution times of Cucumber feature tests at different phases }
    \label{tab:cucumber_times}
\end{table}



  \section{Execution time}
  
All execution times below are mean values of multiple runs and excluding
start-up time. Execution times was measured on a 13" mid-2012 Macbook
Air\footnote{\url{http://www.everymac.com/systems/apple/macbook-
air/specs /macbook-air-core-i5-1.8-13-mid-2012-specs.html}} with 1.83
GHz Intel Core i5 processor and Mac OS 10.9.2. Google Chrome 34.0.1847
was used as browser for execution of browser tests and Jasmine unit
tests. The browser window was focused in order to mitigate effects of
Mac OS X power saving features.\\

\subsubsection{Before the case study}

The total running time of the 59 RSpec integration- and unit tests
before the case study was 23.0 seconds. Average test execution time was
340 ms per test, which means an execution rate of 3.0 tests per
second.\\

Twelve Cucumber browser tests existed at this time (with 178 steps), but
none of these tests passed at this stage. After fixing the most critical
issues with these tests, 36 steps passed in 42.7 seconds. The average
time per test was 3.6 seconds. These numbers are however not entirely
fair compared to later test runs, since large parts of the tests could
not be run at this stage.\\


\subsubsection{After refactoring old tests}

Several of the previous tests was removed and replaced by more efficient
tests during the second part of the case study. 58 RSpec integration-
and unit tests existed after this stage, an the total execution time was
8.3 seconds.\\

The running time of the ten Cucumber tests (with 118 steps) was 160
seconds, and the average time per test was 18 seconds.\\


\subsubsection{After implementation of new functionality}

In total, 81 RSpec integration- and unit tests existed at this stage,
and the execution time of these was 7.6 seconds. Average test execution
time was 94 ms per test, and the rate was 10.7 tests per
second.\\

At this phase, 34 Jasmine unit tests had been written and the total
running time of these was 0.043 seconds. Average execution time was 1.3
ms per test and the rate was 791 tests per second.\\

The running time of the three RSpec browser tests was 12.4 seconds, with
an average time per test of 4.1 seconds per test. One of the Cucumber
tests had been re-factored into RSpec browser tests, and the running
time of the remaining nine Cucumber browser tests (with 111 steps) was
153 seconds. The average time per test was 17 seconds.\\


\begin{table}[t]
    \centering
    \begin{tabular}{l l l l l}
        Phase & No. of tests & Total time (s) & Time per test (ms) & Tests per second\\
        \hline
        Before case-study & 59 & 19.9& 340 & 3.0 \\
        After first part  & 58 & 8.3 & 140 & 6.9 \\
        After second part & 81 & 7.6  & 153 & 10.7\\
    \end{tabular}
    \caption{ Execution times of RSpec integration- and unit tests at different phases }
    \label{tab:unit_times}
\end{table}


\begin{table}[t]
    \centering
    \begin{tabular}{l l l l l}
        Phase & No. of tests & Total time (s) & Time per test (ms) & Tests per second \\
        \hline
        Before case-study & 12 & 43 & 3.6 & 0.27 \\
        After first part  & 10 & 160 & 18 & 0.056\\
        After second part & 9 &  153 & 17 & 0.056\\
    \end{tabular}
    \caption{ Execution times of Cucumber feature tests at different phases }
    \label{tab:cucumber_times}
\end{table}


  \section{Readability and ease of writing}
  \input{results/readability.tex}


\chapter{Discussion}
\label{chap:discussion}

  This chapter discusses the results. (wip)


  \section{Experiences of test-driven development}

  \section{Test efficiency}
  
There are several metrics for evaluating different quality factors of
written tests. One key property of tests is that they must be able to
detect defects in the code, since that is typically the very main reason
for writing tests at all. We have chosen to call this property
\emph{test efficiency}. Another quality factor is the performance of the
tests, i.e. the execution time of the test suite. This section explains
the properties of these quality factors, as well as their purpose.\\

Readability and ease of writing of tests is another important property
which are not mentioned in this section, since it is hard to measure in
an objective way. It also often depends on the testing framework rather
than on the tests themselves. We evaluate these properties for each
chosen testing framework instead.\\

\subsection{Test coverage}
\label{sec:coverage}
\label{sec:results_coverage}

A quick overview of the results are presented in
\fref{tab:unit_coverage}. More information is presented in the following
sections. As discussed in \fref{sec:coverage_frameworks}, we were unable
to find any tool for analyzing anything else than statement test
coverage for Ruby.\\

\begin{table}[t]
    \centering
    \begin{tabular}{l l l}
        Phase & No. of tests & Test coverage\\
        \hline
        Before case-study &       59 & 42.24 \%\\
        After first part  &       58 & 55.25 \%\\
        After second part &       81 & 62.34 \%\\
    \end{tabular}
    \caption{ Statement test coverage of RSpec unit- and integration tests at different phases. }
    \label{tab:unit_coverage}
\end{table}

\begin{table}[t]
    \centering
    \begin{tabular}{l l l}
        Description & No. of tests & Test coverage\\
        \hline
        RSpec browser tests &     3 & 46.3 \%\\
        Cucumber browser tests &  8 & 61.2 \%\\
        All RSpec tests &        84 & 66.8 \%\\
    \end{tabular}
    \caption{ Statement test coverage including browser tests after the second part of the case study. }
    \label{tab:browser_coverage}
\end{table}

\subsubsection{Before the case study}

Before the start of this master's thesis, the average statement coverage
of all RSpec unit- and integration tests was 42 \%.\\

For the client-side code, no unit-tests existed. The test coverage was
thus zero at this state.\\


\subsubsection{After the first part of the case study}

The first part of the case study (described in \fref{sec:casestudy_1})
was focused on rewriting broken tests, since some of the existing tests
was not functional. During this period, a large part of the existing
test suite was either fixed or completely rewritten. Many of the
existing unit- and integration tests was rewritten and a few large
acceptance-level test was replaced by more fine-grained integration
tests. The statement coverage of the increased to 53 \% after this
part of the case study.\\

There were still no client-side unit-tests after this part, since the
focus was fixing the existing server-side tests. The client-side test
coverage was still zero at this state \\

\subsubsection{After the second part of the case study}
\label{sec:result_coverage_end}

The second part of the case study (described in \fref{sec:casestudy_2})
was focused on implementing new functionality while re-factoring old
parts as needed and write tests for new as well as re-factored code. The
first sprint of this part was focused on basic functionality, while the
second sprint was focused on extending and generalizing the new
functionality.\\

When the implementation of the new functionality was finished, statement
coverage of unit- and integration tests for the server side was 62 \%. A
subjective measure indicated that new and re-factored functions in
general has high statement coverage. In cases where full statement
coverage is not achieved, the reason is generally special cases. The
most common example is when error-messages are given when request
parameters are missing or invalid.\\

Statement test coverage for the client-side was 21 \% and the
corresponding branch coverage was 10 \%. A subjective measure
indicates that almost all of the newly implemented functions achieves
full statement coverage. The branch coverage is in general also high for
newly implemented functions, but conditionals for special cases (such as
when variables are zero or not set) are sometimes not covered.\\

\subsubsection{Test coverage for browser tests}

All metrics in the previous sections refer to test coverage for unit-
and integration tests. As one of the last steps of the second part of
the case study, browser tests was added in order to do system-level
testing.\\

The total statement test coverage for the RSpec browser tests alone was
46 \%, and the statement test coverage for all Ruby tests was 67 \%.
With the tools used, it was not possible to measure Javascript test
coverage for the test suite including browser tests.\\

The total statement test coverage for the Cucumber browser tests alone
was 61 \%. It was not possible to calculate the total coverage for
Cucumber and RSPec tests combined.\\


\subsection{Mutation testing}
\label{sec:theory_mutation}
An alternative to draw conclusions from which paths of the code that is
run by a test, as done when using test coverage, is to draw conclusions
from what happens when we modify the code. The idea is that if the code
if incorrect, the test should fail. Thus, we can modify the code so it
becomes incorrect and then look at whether the test fails or not.\\

Mutation testing is done by creating several versions of the tested
code, and introduce slight modifications into each one of them. Each
such version containing a mutated version of the original source code is
called a \emph{mutant}. All tests which we want to evaluate are run for
each mutant. If at least one of the tests fails, the mutant is
considered to be \emph{killed}. We can measure the efficiency of the
test suite as the ratio of killed mutants versus the total number of
mutants.\cite{wiki:mutation}\\


\begin{lstlisting}[caption=Example of a piece of code before mutation,
                   label=lst:mutation_before, float=t]
    def odd?(x, y)
        (x % 2) && (y % 2)
    end
\end{lstlisting}


\begin{lstlisting}[caption=Two mutated versions of \ref{lst:mutation_before},
                   label=lst:mutant_1, float=t]
    def odd?(x, y)
        (x % 2) && (x % 2)
    end

    def odd?(x, y)
        (x % 2) || (y % 2)
    end
\end{lstlisting}


\subsection{Execution time}

All execution times below are mean values of multiple runs and excluding
start-up time. Execution times was measured on a 13" mid-2012 Macbook
Air\footnote{\url{http://www.everymac.com/systems/apple/macbook-
air/specs /macbook-air-core-i5-1.8-13-mid-2012-specs.html}} with 1.83
GHz Intel Core i5 processor and Mac OS 10.9.2. Google Chrome 34.0.1847
was used as browser for execution of browser tests and Jasmine unit
tests. The browser window was focused in order to mitigate effects of
Mac OS X power saving features.\\

\subsubsection{Before the case study}

The total running time of the 59 RSpec integration- and unit tests
before the case study was 23.0 seconds. Average test execution time was
340 ms per test, which means an execution rate of 3.0 tests per
second.\\

Twelve Cucumber browser tests existed at this time (with 178 steps), but
none of these tests passed at this stage. After fixing the most critical
issues with these tests, 36 steps passed in 42.7 seconds. The average
time per test was 3.6 seconds. These numbers are however not entirely
fair compared to later test runs, since large parts of the tests could
not be run at this stage.\\


\subsubsection{After refactoring old tests}

Several of the previous tests was removed and replaced by more efficient
tests during the second part of the case study. 58 RSpec integration-
and unit tests existed after this stage, an the total execution time was
8.3 seconds.\\

The running time of the ten Cucumber tests (with 118 steps) was 160
seconds, and the average time per test was 18 seconds.\\


\subsubsection{After implementation of new functionality}

In total, 81 RSpec integration- and unit tests existed at this stage,
and the execution time of these was 7.6 seconds. Average test execution
time was 94 ms per test, and the rate was 10.7 tests per
second.\\

At this phase, 34 Jasmine unit tests had been written and the total
running time of these was 0.043 seconds. Average execution time was 1.3
ms per test and the rate was 791 tests per second.\\

The running time of the three RSpec browser tests was 12.4 seconds, with
an average time per test of 4.1 seconds per test. One of the Cucumber
tests had been re-factored into RSpec browser tests, and the running
time of the remaining nine Cucumber browser tests (with 111 steps) was
153 seconds. The average time per test was 17 seconds.\\


\begin{table}[t]
    \centering
    \begin{tabular}{l l l l l}
        Phase & No. of tests & Total time (s) & Time per test (ms) & Tests per second\\
        \hline
        Before case-study & 59 & 19.9& 340 & 3.0 \\
        After first part  & 58 & 8.3 & 140 & 6.9 \\
        After second part & 81 & 7.6  & 153 & 10.7\\
    \end{tabular}
    \caption{ Execution times of RSpec integration- and unit tests at different phases }
    \label{tab:unit_times}
\end{table}


\begin{table}[t]
    \centering
    \begin{tabular}{l l l l l}
        Phase & No. of tests & Total time (s) & Time per test (ms) & Tests per second \\
        \hline
        Before case-study & 12 & 43 & 3.6 & 0.27 \\
        After first part  & 10 & 160 & 18 & 0.056\\
        After second part & 9 &  153 & 17 & 0.056\\
    \end{tabular}
    \caption{ Execution times of Cucumber feature tests at different phases }
    \label{tab:cucumber_times}
\end{table}



  \section{Test execution time}
  
All execution times below are mean values of multiple runs and excluding
start-up time. Execution times was measured on a 13" mid-2012 Macbook
Air\footnote{\url{http://www.everymac.com/systems/apple/macbook-
air/specs /macbook-air-core-i5-1.8-13-mid-2012-specs.html}} with 1.83
GHz Intel Core i5 processor and Mac OS 10.9.2. Google Chrome 34.0.1847
was used as browser for execution of browser tests and Jasmine unit
tests. The browser window was focused in order to mitigate effects of
Mac OS X power saving features.\\

\subsubsection{Before the case study}

The total running time of the 59 RSpec integration- and unit tests
before the case study was 23.0 seconds. Average test execution time was
340 ms per test, which means an execution rate of 3.0 tests per
second.\\

Twelve Cucumber browser tests existed at this time (with 178 steps), but
none of these tests passed at this stage. After fixing the most critical
issues with these tests, 36 steps passed in 42.7 seconds. The average
time per test was 3.6 seconds. These numbers are however not entirely
fair compared to later test runs, since large parts of the tests could
not be run at this stage.\\


\subsubsection{After refactoring old tests}

Several of the previous tests was removed and replaced by more efficient
tests during the second part of the case study. 58 RSpec integration-
and unit tests existed after this stage, an the total execution time was
8.3 seconds.\\

The running time of the ten Cucumber tests (with 118 steps) was 160
seconds, and the average time per test was 18 seconds.\\


\subsubsection{After implementation of new functionality}

In total, 81 RSpec integration- and unit tests existed at this stage,
and the execution time of these was 7.6 seconds. Average test execution
time was 94 ms per test, and the rate was 10.7 tests per
second.\\

At this phase, 34 Jasmine unit tests had been written and the total
running time of these was 0.043 seconds. Average execution time was 1.3
ms per test and the rate was 791 tests per second.\\

The running time of the three RSpec browser tests was 12.4 seconds, with
an average time per test of 4.1 seconds per test. One of the Cucumber
tests had been re-factored into RSpec browser tests, and the running
time of the remaining nine Cucumber browser tests (with 111 steps) was
153 seconds. The average time per test was 17 seconds.\\


\begin{table}[t]
    \centering
    \begin{tabular}{l l l l l}
        Phase & No. of tests & Total time (s) & Time per test (ms) & Tests per second\\
        \hline
        Before case-study & 59 & 19.9& 340 & 3.0 \\
        After first part  & 58 & 8.3 & 140 & 6.9 \\
        After second part & 81 & 7.6  & 153 & 10.7\\
    \end{tabular}
    \caption{ Execution times of RSpec integration- and unit tests at different phases }
    \label{tab:unit_times}
\end{table}


\begin{table}[t]
    \centering
    \begin{tabular}{l l l l l}
        Phase & No. of tests & Total time (s) & Time per test (ms) & Tests per second \\
        \hline
        Before case-study & 12 & 43 & 3.6 & 0.27 \\
        After first part  & 10 & 160 & 18 & 0.056\\
        After second part & 9 &  153 & 17 & 0.056\\
    \end{tabular}
    \caption{ Execution times of Cucumber feature tests at different phases }
    \label{tab:cucumber_times}
\end{table}


  \section{Future work}
  This chapter discusses what could be done on the subject in a future
study.



\chapter{Conclusions}
\label{chap:conclusions}

    
This chapter presents some final thoughts and reflections on the results
and summarizes the previously discussed conclusions. We also present a
summary of our proposed solution for testing a web application.\\

\section{General conclusions}

\begin{itemize}

\item  Working with the chosen testing frameworks (RSpec and Jasmine
with the Karma test runner) has worked very well for testing this
particular application. We believe that it would also work very well for
testing most similar web applications.\\

\item The experience of using test-driven development as well as some
topics originating from behavior-driven development has been pleasant.
While this is a subjective opinion, we believe that this might be the
case for many other software developers and for other projects as well.
We have also found that the newly implemented functionality is well
tested and that its test efficiency is high, which may be another
benefit from using these methodologies. Writing tests using a
ubiquitous language is however not beneficial for this particular
application.\\

\item The combination of many integration- and unit-tests, complemented
by a few browser tests was successful for this project. It might however
be hard to determine the best level of testing for certain
functionality.\\

The level of testing and the combination of different kinds of tests
basically depends on the application. It is neither possible to say that
writing higher-level unit- or integration- test is generally worse
than writing lower-level unit tests, nor the contrary. Using browser
tests as the primary testing method for most projects is probably not
the best solution, however.\\

\item A test suite can speed up significantly by using factory objects
rather than using fixtures or manually create data when the test suite
is initialized. While these tests still are not in the same magnitude of
speed as the low-level unit tests written for the client side, they are
still fast enough to be usable for using test-driven development
methodologies.\\

\item Using metrics for test efficiency such as test coverage is usable
for finding parts of the code that lacks testing, in order to make tests
better. We believe that statement test coverage is usable, even if
branch coverage is more helpful and returns more fair coverage
percentages. The increased test efficiency could possibly lead to
finding more software defects due to more efficient tests, but we have
not found any defects of this type by using test efficiency metrics in
this project.\\

\end{itemize}


\section{Suggested solution for testing a web application}

For a web application with at least some amount of client-side and
server-side code, we would recommend unit-testing for the client as well
as the server. In addition, system-level browser tests should be used.
Most of the written tests should be on unit- or integration- level
tests, but the exact proportions depend on the application.\\

We would recommend applications that use Javascript or CoffeeScript on
the client side to use the Jasmine testing framework together with the
Karma test runner. For testing the client-side in Rails projects,
Teaspoon could be an alternative to Karma. See \fref{sec:js_test} for
more discussion on the subject.\\

A server side written Ruby on Rails could be tested using RSpec, with
factory\_girl for generating test data. Controllers can preferably be
tested using higher-level tests, while model instance methods often can
be tested using lower-level unit tests. See \fref{sec:ruby_test} for
more details on Rails testing frameworks.\\

Selenium is highly useful for system-level tests. Using a higher-level
framework such as Capybara rather than using Selenium directly is
recommended. The page object pattern should be used. SitePrism is useful
for this purpose. See \fref{sec:choices_browser} for more information
about these frameworks.\\



\newpage

\chapter*{References}
\begin{multicols}{2}
    \small
    \renewcommand{\bibsection}{ \vspace{-\baselineskip}\vspace{-1.1mm} }
    \bibliographystyle{plainnat}
    \bibliography{references}
\end{multicols}

\end{document}
