
As discussed in \fref{sec:coverage_frameworks}, we were unable to find any tool
for analyzing anything else than statement test coverage for Ruby.\\

\subsubsection{Before the case study}

Before the start of this master's thesis, the average statement coverage
of all RSpec unit- and integration tests was 45.2 \%. A major part of
the modules with high test coverage score was modules without any
methods, such as data models which only contain model definitions. Such
modules typically get full test coverage just by being loaded by the
application. Since these modules also can have methods which needs
tests, we could not just exclude all such modules either.\\

For the client-side code, no unit-tests existed. The test coverage was
thus zero at this state.\\


\subsubsection{After the first part of the case study}

The first part of the case study (described in \fref{sec:casestudy_1})
was focused on rewriting broken tests, since some of the existing tests
was not functional. During this period, a large part of the existing
test suite was either fixed or completely rewritten. Many of the
existing unit- and integration tests was rewritten and a few large
acceptance-level test was replaced by more fine-grained integration
tests. The statement coverage increased to 59.5 \% after this part of
the case study.


\subsubsection{After the second part of the case study}

The second part of the case study (described in \fref{sec:casestudy_2})
was focused on implementing new functionality while re-factoring old
parts as needed and write tests for new as well as re-factored code. The
first sprint of this part was focused on basic functionality, while the
second sprint was focused on extending and generalizing the new
functionality.\\

When the implementation of the new functionality was finished, statement
coverage of unit- and integration tests was 69.5 \%.
