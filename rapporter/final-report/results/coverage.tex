
As discussed in \fref{sec:coverage_frameworks}, we were unable to find
any tool for analyzing anything else than statement test coverage for
Ruby. Since most of the implemented code consisted of extensions or
refactoring to existing files, it is hard to get a specific metric value
for how well tested the modified parts of the software is. This is
because these metrics are given in percent per file, and since the same
file contains a lot of unmodified code, that will affect the result. We
will instead use a subjective measurement based on the detailed coverage
report output for estimating test coverage of new functionality.\\


\subsubsection{Before the case study}

Before the start of this master's thesis, the average statement coverage
of all RSpec unit- and integration tests was 42 \%. A major part of
the modules with high test coverage score was modules without any
methods, such as data models which only contain model definitions. Such
modules typically get full test coverage just by being loaded by the
application. Since these modules also can have methods which needs
tests, we could not just exclude all such modules either.\\

For the client-side code, no unit-tests existed. The test coverage was
thus zero at this state.\\


\subsubsection{After the first part of the case study}

The first part of the case study (described in \fref{sec:casestudy_1})
was focused on rewriting broken tests, since some of the existing tests
was not functional. During this period, a large part of the existing
test suite was either fixed or completely rewritten. Many of the
existing unit- and integration tests was rewritten and a few large
acceptance-level test was replaced by more fine-grained integration
tests. The statement coverage of the increased to 53 \% after this
part of the case study.\\

There were still no client-side unit-tests after this part, since the
focus was fixing the existing server-side tests. The client-side test
coverage was still zero at this state \\

\subsubsection{After the second part of the case study}

The second part of the case study (described in \fref{sec:casestudy_2})
was focused on implementing new functionality while re-factoring old
parts as needed and write tests for new as well as re-factored code. The
first sprint of this part was focused on basic functionality, while the
second sprint was focused on extending and generalizing the new
functionality.\\

When the implementation of the new functionality was finished, statement
coverage of unit- and integration tests for the server side was 62 \%.\\

Statement test coverage for the client-side was 21 \% and the
corresponding branch coverage was 10 \%. A subjective measure
indicates that almost all of the newly implemented functions achieves
full statement coverage. The branch coverage is in general also high for
newly implemented functions, but conditionals for special cases (such as
when variables are zero or not set) are sometimes not covered.\\


\begin{table}[t]
    \centering
    \begin{tabular}{l l}
        Phase & Test coverage\\
        \hline
        Before case-study & 42.24 \%\\
        After first part  & 55.25 \%\\
        After second part & 62.34 \%\\
    \end{tabular}
    \caption{ Statement test coverage of RSpec unit- and integration tests at different phases. }
    \label{tab:unit_coverage}
\end{table}
