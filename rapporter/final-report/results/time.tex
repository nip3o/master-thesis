\label{sec:results_time}

All execution times below are mean values of multiple runs and excluding
start-up time. Execution times are those reported by each testing tool
when run on a 13" mid-2012 Macbook
Air\footnote{\url{http://www.everymac.com/systems/apple/macbook-
air/specs /macbook-air-core-i5-1.8-13-mid-2012-specs.html}} with 1.83
GHz Intel Core i5 processor and Mac OS 10.9.2. Google Chrome 34.0.1847
was used as browser for execution of browser tests and Jasmine unit
tests. The browser window was focused in order to neutralize effects of
Mac OS X power saving features.\\

A quick overview of the results is presented in \fref{tab:unit_times},
\ref{tab:cucumber_times}, \ref{tab:jasmine_times}, and
\ref{tab:rspec_browser_times}. The same results are also presented as
text in the following sections.\\

\begin{table}[t]
    \centering
    \begin{tabular}{l l l l l}
        Phase & No. of tests & Total time & Time per test & Tests per second\\
        \hline
        Before the case study & 59 & 19.9 s & 340 ms & 3.0 \\
        After the first part  & 58 & 8.3 s  & 140 ms & 6.9 \\
        After the second part & 81 & 7.6 s  & 153 ms & 10.7\\
    \end{tabular}
    \caption{ Execution times of RSpec integration- and unit tests at different phases }
    \label{tab:unit_times}
\end{table}

\begin{table}[t]
    \centering
    \begin{tabular}{l l l l l}
        Phase & No. of tests & Total time & Time per test & Tests per second \\
        \hline
        Before the case study & 12 & 43 s & 3.6 s & 0.27 \\
        After the first part  & 10 & 160 s & 18 s & 0.056\\
        After the second part & 8 &  155 s & 17 s & 0.056\\
    \end{tabular}
    \caption{ Execution times of Cucumber feature tests at different phases }
    \label{tab:cucumber_times}
\end{table}

\begin{table}[t]
    \centering
    \begin{tabular}{l l l l l}
        Phase & No. of tests & Total time & Time per test & Tests per second \\
        \hline
        After second part & 3 &  12.4 s & 4.1 s & 0.24\\
        At latest revision & 10 & 38.2 s & 3.8 & 0.26\\
    \end{tabular}
    \caption{ Execution times of RSpec browser tests }
    \label{tab:rspec_browser_times}
\end{table}

\begin{table}[t]
    \centering
    \begin{tabular}{l l l l l}
        Phase & No. of tests & Total time & Time per test & Tests per second \\
        \hline
        After second part & 34 &  0.043 s & 1.3 ms & 791\\
    \end{tabular}
    \caption{ Execution times of Jasmine unit tests }
    \label{tab:jasmine_times}
\end{table}


\subsection{Before the case study}

The total running time of the 59 RSpec integration- and unit tests
before the case study was 23.0 seconds. Average test execution time was
340 ms per test, which means an execution rate of 3.0 tests per
second.\\

Twelve Cucumber browser tests existed at this time (with 178 steps), but
none of these tests passed at this stage. After fixing the most critical
issues with these tests, 36 steps passed in 42.7 seconds. The average
time per test was 3.6 seconds.\\


\subsection{After refactoring old tests}

Several of the previous tests was removed and replaced by more efficient
tests during the second part of the case study. 58 RSpec integration-
and unit tests existed after this stage, an the total execution time was
8.3 seconds.\\

The running time of the ten Cucumber tests (with 118 steps) was 160
seconds, and the average time per test was 18 seconds.\\


\subsection{After implementation of new functionality}

In total, 81 RSpec integration- and unit tests existed at this stage,
and the execution time of these was 7.6 seconds. Average test execution
time was 94 ms per test, and the rate was 10.7 tests per
second.\\

At this phase, 34 Jasmine unit tests had been written and the total
running time of these was 0.043 seconds. Average execution time was 1.3
ms per test and the rate was 791 tests per second.\\

Two of the Cucumber tests had been refactored into RSpec browser tests
at this stage, and the running time of the remaining eight Cucumber
browser tests (with 111 steps) was 155 seconds. The average time per
test was 19 seconds.\\

The running time of the three newly implemented RSpec browser tests was
12.4 seconds, with an average time of 4.1 seconds per test. After the
case study, other developers added additional RSpec browser tests.  At
the latest revision of the code as of May 2014, there are ten RSpec
browser tests, and the total running time of these are 38 seconds. The
average time per test is 3.8 seconds. While these new tests are not
strictly a part of the case study, we have chosen to include them in
order to get more robust measurements (since we get a sample size of ten
instead of three).\\

