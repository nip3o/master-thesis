
Just like Cucumber, RSpec also claims to be made for use with BDD. In
contrast, it only uses descriptive strings instead of a full ubiquitous
language and can be used for writing isolated unit-tests as well as
integration- and browser tests.\\

We found it quite straightforward to write tests with RSpec, and to use
descriptive strings rather than function names for describing tests. The
plug-in rspec-mocks\footnote{\url{https://github.com/rspec/rspec-mocks}}
was used in some situations where mocking was required, since the RSpec
core does not include support for this. One major drawback of rspec-
mocks is that it allows stubbing non-existent methods and properties,
which as discussed in \fref{sec:theory_mocks} can be dangerous. We
worked around this by writing a helper for checking existence of
properties before stubbing.\\

\begin{lstlisting}[caption=Example of RSpec tests for a module.,
                   label=lst:rspec, float=t]
describe Math do
  describe '#minus' do
    it 'returns the difference between two positive integers' do
      expect(Math::minus(3, 1)).to eq(2)
    end

    it 'returns the sum if the second integer is negative' do
      expect(Math::minus(5, -2)).to eq(7)
    end
  end

  describe '#plus' do
    it 'returns the sum of two positive integers' do
      expect(Math::plus(1, 2)).to eq(3)
    end
  end
end
\end{lstlisting}
