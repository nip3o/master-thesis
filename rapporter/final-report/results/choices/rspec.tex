
Just like Cucumber, RSpec also claims to be made for use with
BDD\cite{web:rspec}. In contrast, it only uses descriptive strings
instead of a full ubiquitous language and can be used for writing
isolated unit-tests as well as integration- and browser tests. \\

As an alternative to RSpec, we also considered
Minitest\footnote{\url{https://github.com/seattlerb/minitest}}. Similar
to RSpec, Minitest is popular as well as actively developed. In contrast
to RSpec, it is more modular and claims to be more readable. It also
claims to be more minimalistic and lightweight. By looking at code
examples and documentation, we however found that the syntax of the
tests for recent versions of both these frameworks seemed to be very
similar. We did not find any considerable advantages of using Minitest
over RSpec and therefore concluded that RSpec was a better option, since
some of the existing tests was written using this framework.\\

We found it quite straightforward to write tests with RSpec, and to use
descriptive strings rather than function names for describing tests as
seen in code listing \ref{lst:rspec}. The plug-in rspec-
mocks\footnote{\url{https://github.com/rspec/rspec-mocks}} were used in
some situations where mocking was required, since the RSpec core does
not include support for this. One major drawback of rspec- mocks is that
it allows stubbing non-existent methods and properties, which as
discussed in \fref{sec:theory_mocks} can be dangerous. We worked around
this by writing a helper for checking existence of properties before
stubbing them.\\

\begin{lstlisting}[caption=Example of RSpec tests for a module.,
                   label=lst:rspec, float=t]
describe Math do
  describe '#minus' do
    it 'returns the difference between two positive integers' do
      expect(Math::minus(3, 1)).to eq(2)
    end

    it 'returns the sum if the second integer is negative' do
      expect(Math::minus(5, -2)).to eq(7)
    end
  end

  describe '#plus' do
    it 'returns the sum of two positive integers' do
      expect(Math::plus(1, 2)).to eq(3)
    end
  end
end
\end{lstlisting}
