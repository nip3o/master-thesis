
\label{sec:js_test}

There are a few different frameworks for testing Javascript or
CoffeeScript code. We had previously good experiences from working with
Jasmine\footnote{\url{http://jasmine.github.io/}}. We also found that
this framework seemed to be very popular, actively developed and had
good documentation\footnote{It is worth to mention that the Jasmine
documentation is basically its own test suite with some additional
comments. This works incredibly good in this case, presumably since it
is a testing framework and the tests are very well-written.}. Code
listing \ref{lst:jasmine} shows an example of a Jasmine test written in
CoffeeScript. The syntax of Jasmine is inspired by RSpec, which is
another advantage since RSpec is used for the server side tests.\\

\subsubsection{Test runner}

The Jasmine framework provides a way of writing tests, but a test runner
is also required in order to run the tests. Jasmine ships with a basic
test runner, which was used initially and worked natively using
the Jasmine Ruby gem. A screenshot of this test runner is shown in
\fref{fig:jasmine_runner}.\\

\begin{figure}
\centering
\includegraphics[width=0.8\textwidth]{results/choices/jasmine_runner}
\caption{The test runner bundled with Jasmine.}
\label{fig:jasmine_runner}
\end{figure}

The Jasmine test runner did however have multiple issues. First of all,
it runs completely in the browser. Switching to the browser and reload
the page in order to run the tests are not excessively problematic, but
might be tiresome in the long run when using a test-driven development
methodology.\\

A larger problem with the default test runner was that the asset
handling, i.e. the process of compiling CoffeeScript into Javascript.
Rails handles this compilation upon each reload when using the Jasmine
gem. Since the used version of Rails re-compiles all assets upon page
load if any file has been changed, this process takes a while, which
means that each test run could take up to 10-15 seconds even though the
actual tests only takes a fraction of a second to run. The asset
compilation also got stuck for apparently no reason once in a while.
Since the server port on which the test runner runs cannot be specified,
it is also impossible to restart the test runner without manually
killing its system process.\\

Another issue with the Jasmine test runner is that syntax errors are
printed in the terminal rather than in the browser window where the test
result is reported, which is confusing. In practice, syntax errors were
often undetected for a long time, which required more debugging than
necessary.\\

Due to these issues, we switched to the Karma\footnote{\url{http
://karma-runner.github.io/}} test runner. Karma originates from a
master's thesis by \citet{article:karma}, which covers several problems
with the Jasmine test runner as well as with some other Javascript test
runners. Karma was designed to solve several of these issues and to be
used with test-driven software methodologies. Tests are run in a browser
as soon as a file is changed, and the results are reported back to the
terminal and displayed as shown in \fref{fig:karma_runner}. Re-
compilation of source files and tests is very fast and we did not
experience any stability issues. One minor drawback is however that
pending tests is not displayed very clearly. This can be solved by using
a different Karma test reporter.\\

It took some effort getting Karma to work with Rails, since Karma is
written in Node.js and does not have any knowledge about which
CoffeeScript files that exist in the Rails application. The task of
finding the location of all such files became more complex since
external Javascript libraries such as jQuery was loaded as Ruby gems,
and therefore not even located inside the project folder. We ended up
using a slightly modified version of a Rake script from a blog entry by
\citet{web:saunier_angular} for bootstrapping Karma in a Rails
environment. This script basically collects a list of filenames for all
Javascript assets by using Rails internal modules for asset handling,
and injects it into Karma's configuration file.\\

After the case study, we discovered another Javascript test runner
called Teaspoon\footnote{\url{https://github.com/modeset/teaspoon}},
which is built for Rails and can discover assets by default. We believe
that this test runner looks very promising and may be more suitable for
Rails projects than Karma. At the time of this writing, Teaspoon does
not have support for the most recent version of Jasmine and therefore
does not work with our test suite. We were thus unable to evaluate
Teaspoon any further.\\

\begin{lstlisting}[caption=Example of Jasmine tests for a module (compare with code listing \ref{lst:rspec}).,
                   label=lst:jasmine, float=t, language=HTML]
describe 'Math', ->
  describe 'minus', ->
    it 'returns the difference between two positive integers', ->
      expect(Math.minus(3, 1)).toEqual(2)

    it 'returns the sum if the second integer is negative', ->
      expect(Math.minus(5, -2)).toEqual(7)

  describe 'plus', ->
    it 'returns the sum of two positive integers', ->
      expect(Math.plus(1, 2)).toEqual(3)
\end{lstlisting}

\begin{figure}
\centering
\includegraphics[width=0.8\textwidth]{results/choices/karma_runner}
\caption{The Karma test runner.}
\label{fig:karma_runner}
\end{figure}
