
We worked with Cucumber during the first part of the case study, since
the major part of all existing tests was written using this framework.
Cucumber is a framework for acceptance-level testing using BDD
methodology. Tests, called \emph{features}, are written using an
ubiquitous language as seen in listing \ref{lst:cucumber_feature}. The
action for each line, called \emph{step}, of the feature is specified
using a \emph{step definition}, as seen in listing
\ref{lst:cucumber_step}.\cite{web:cucumber}\\

One benefit from using Cucumber is that all steps are reusable, which
means that code duplication can be avoided. However, it can be hard to
write the steps in such way that they benefit from this, and it
sometimes also requires a lot of parameters to be passed in to each
step. Cucumber also provides code snippets for creating step
definitions, which evades some unnecessary work.\\

Apart from these benefits, we found Cucumber tiresome to work with. The
separation between features and step definitions makes it hard to get an
overview of the code executed during the test, and it is often hard to
find specific step definitions. We also experienced problems with the
mapping step definitions, where the generated step definition simply did
not match the written step.\\

The chosen level of testing is another big issue. We felt that using
TDD-methodology was hard to do with system-level tests since they take
long time to execute and affects a much larger part of the software than
the part we are typically working with when implementing new
functionality. Considering these drawbacks, we decided to not continue
the use of Cucumber for this thesis.\\


\begin{lstlisting}[caption=Example of a Cucumber test.,
                   label=lst:cucumber_feature, float=t, language=HTML]
Feature: creating new cookies
  As a bakery worker
  So that I can sell cookies to my customers
  I want to create a new cookie object

Background:
  Given a user and a cookie type called "Chocolate chip"
  And I have created one cookie

Scenario: create a new cookie
  When I visit the page for creating cookies
  Then I should see one cookie
  When I create a new Chocolate chip cookie
  Then I should see two cookies
\end{lstlisting}


\begin{lstlisting}[caption=Cucumber step definition for the step on row 13 in code listing \ref{lst:cucumber_feature}.,
                   label=lst:cucumber_step, float=t]
When /^I create a new (.+) cookie$/ do |cookie_type|
  # Code for creating a new cookie
end
\end{lstlisting}
