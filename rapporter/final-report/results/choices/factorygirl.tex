\MakeShortVerb{\|}

One important framework used was
factory\_girl\footnote{\url{https://github.com/thoughtbot/factory\_girl}},
which is used for generating factory objects. As discussed in
\fref{sec:theory_mocks}, factory objects behaves just like instances of
model objects, but has several advantages. As alternatives to
factory\_girl, we also considered
Machinist\footnote{\url{https://github.com/notahat/machinist}} and
Fabrication\footnote{\url{http://www.fabricationgem.org/}}. Machinist
was discarded since it is no longer actively developed. We did not find
any significant differences between Fabrication and factory\_girl, and
therefore chose the latter since it was more popular.\\

There are many advantages of using a framework such as factory\_girl
rather than just instantiate model objects by hand (i.e. just write
|MyModel.new| in Ruby to create a new model instance). First of all,
factory\_girl automatically passes in default values for required
parameters, so that we only need to supply the attributes needed in a
particular test. Secondly, related objects are also created
automatically, which typically saves as huge amount of work compared to
manual creation of objects. Code listing \ref{lst:factory_def} and
\ref{lst:factory_use} shows a factory definition and its usage. The name
of the cookie and a new Bakery-object is created automatically since we
do not give them as parameters when using the factory.\\

We did initially have some issues with the creation of related objects
since the factory\_girl documentation did not cover working with
document-based databases such as MongoDB in our case, but we eventually
found out how to do this.\\

One feature that we felt was missing in factory\_girl was the ability to
specify attributes on related objects, for example to specify the name
of a Bakery when creating a new Cookie. It is of course possible to
first create a Bakery object and then creating a Cookie and pass in the
created bakery object, but a shortcut for doing this would have been
convenient in some situations. To our knowledge, Fabrication also lacks
this feature.\\

\begin{lstlisting}[caption=Example usage of the factory defined in code listing \ref{lst:factory_def}.,
                   label=lst:factory_use, float=t]
FactoryGirl.create :cookie, diameter: 4.5, thickness: 0.5
\end{lstlisting}

\begin{lstlisting}[caption=A factory definition for a Cookie model.,
                   label=lst:factory_def, float=t]
FactoryGirl.define do
  factory :cookie do
    name      'Vanlilla dream'
    diameter  1
    thickness 2
    bakery    { FactoryGirl.build(:bakery) }
  end
end
\end{lstlisting}

