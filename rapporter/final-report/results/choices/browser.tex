\label{sec:choices_browser}

\subsubsection{Selenium}

The absolutely most widespread framework for browser testing is
Selenium\footnote{\url{http://docs.seleniumhq.org/}}. In fact, we have
not even been able to find any other frameworks for running tests in
real browsers, possibly since it takes a lot of effort to integrate with
a large number of browsers on a large number of platforms. Selenium
supports a number of popular programming languages, for example Java,
C\#, Python and Ruby, and also has support the majority of all modern
web browsers on Windows as well as Linux and Mac OS. The interface used
by newer versions of Selenium\footnote{There are two major versions of
Selenium; Selenium RC and the newer Selenium 2. We have only considered
the latter, which is the only of the two versions that uses WebDriver.}
for interacting with browsers, WebDriver, has been proposed as an W3C
Internet standard \cite{web:webdriver}, which may indicate that support
for Selenium is very likely to continue to be present in future browser
versions. \cite{wiki:selenium}\\

While Selenium is very widespread and seems to be the only option for
running tests in real browsers, its API is on a rather low level. In
order to fill in a string of text to a text field, we have to locate the
label of the field using an XPath\footnote{XPath is a language for
selecting elements and attributes in an XML document, such as an
website.} function, then figure out its associated text element and
sending out text as keystrokes to this element. Rather than writing
helper methods for such functionality ourselves, we decided to use
a higher-level framework.\\

\subsubsection{Capybara}
Capybara\footnote{\url{https://github.com/jnicklas/capybara}} provides a
much higher level API for browser testing compared to Selenium, and
besides from using it with Selenium it can also be used with drivers for
headless browsers or for testing frameworks using mock requests. Tests
can also be written using multiple testing frameworks, such as RSpec
in our case.\\

We did not find any particular difficulties when using Capybara. Its API
provides very convenient high-level helpers for every common task that
we needed, and we did not experience any problems with any of them. An
example of a browser test using Capybara is found in code listing
\ref{lst:capybara}.\\


\begin{lstlisting}[caption=A browser test written in RSpec using Capybara.,
                   label=lst:capybara, float=t]
describe 'creating new cookies' do
  before do
    FactoryGirl.create(:cookie_type, name: 'Chocolate chip')
    login_as(FactoryGirl.create(:user))
  end

  it 'should be possible to create new cookies' do
    visit(new_cookie_path)

    fill_in('Amount', with: 1)
    select('Chocolate chip', from: 'Cookie type')
    click_button('Save')

    expect(page).to(have_content('The cookie has been created.'))
  end
end
\end{lstlisting}


\subsubsection{SitePrism}

As mentioned in \fref{sec:testing_web}, one way of writing more
structured browser tests is to use the page object pattern. We chose to
use this pattern from the start rather than cleaning up overly
complicated tests afterwards. While this pattern is perfectly possible
to use without any frameworks, we found a framework called
SitePrism\footnote{\url{https://github.com/natritmeyer/site_prism}}
that provided some additional convenient functionality.\\

SitePrism makes it possible to define pages of an application by
specifying its URL, as well as elements and sections of a specific page
by specifying a CSS or XPath selector. An example of a page definition
can be seen in code listing \ref{lst:siteprism_page}. Page objects
automatically get basic methods for accessing elements, and allow us to
define additional methods for each page or element. This allows us to
use higher-level functions in our tests rather than locating elements
manually in our Capybara tests. An example test using SitePrism is seen
in code listing \ref{lst:siteprism_test}.\\

One drawback of using SitePrism can be that elements currently must be
specified using selectors rather than having the possibility to select
them based on text, which is possible with Capybara directly. It would
on the other hand be easy to create a method for doing this using the
helper methods provided by Capybara. One could also claim that using
text for locating elements only makes sense for some specific types of
elements, such as buttons and links, not for elements in general.\\

Overall, we found the page object pattern as well as SitePrism very
convenient to work with.\\

\begin{lstlisting}[caption=Page definition for a page with a list of cookie information.,
                   label=lst:siteprism_page, float=t]

class CookieRowSection < SitePrism::Section
  element :name, '.name'
  element :amount, '.amount'
end

class CookiePage < SitePrism::Page
  set_url('/cookies')

  element :heading, 'h1'
  element :new_cookie_button, 'button#createCookie'
  sections :cookies, CookieRowSection, '.cookieInfo'

  def create_cookie
    self.new_cookie_button.click()
  end
end

\end{lstlisting}

\begin{lstlisting}[caption=Browser test using SitePrism a page object defined in
                           code listing \ref{lst:siteprism_page}.,
                   label=lst:siteprism_test, float=t]

describe 'cookie information page' do
  before do
    FactoryGirl.create(:cookie, name: 'Vanilla dream', amount: 2)
    FactoryGirl.create(:cookie, name: 'Apple crush', amount: 4)
    @cookie_page = CookiePage.new
    @cookie_page.load()
  end

  it 'should contain a list of cookies in alphabetical order' do
    expect(page.heading.text).to(eq('List of cookies'))
    expect(page.cookies.length).to(be(2))

    cookie = page.cookies.first
    expect(cookie.name.text).to(eq('Apple crush'))
    expect(cookie.amount.text).to(eq('2'))
  end
end

\end{lstlisting}
