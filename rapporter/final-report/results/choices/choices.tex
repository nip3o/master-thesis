\MakeShortVerb{\|}
\label{sec:choices}

One important questions of this project is to find a suite of relevant
frameworks in order to work with testing in a Rails application, and
gather experience with working with these. This subsection presents our
evaluation of the different frameworks and tools used for testing.\\

The choice of frameworks for development was mainly given by the
constituent, since the existing software was written in Ruby on Rails
with KnockoutJS and a MongoDB database. The main server-side language
was thus Ruby, and the client-side code was written in CoffeeScript.
CoffeeScript is a scripting language which compiles into Javascript.\\

For the choice of testing-related frameworks, we chose to look for
frequently used and active developed open source frameworks.
Technologies that are used by many people intuitively often has more
resources on how they are used, and also has the advantage of being more
likely to be recognized by future developers. Active development of used
frameworks is crucial, most importantly since they are likely to be
incompatible with future versions of other frameworks (such as Rails).
Another benefit is that new features and bug fixes are released. The
Ruby Toolbox website \footnote{\url{https://www.ruby-toolbox.com/}},
which uses information from the Github and RubyGems websites, was
consulted in order to find frameworks with mentioned qualities.\\

\subsection{Ruby test frameworks and tools}

\label{sec:ruby_test}

Before the case study, Cucumber\footnote{\url{http://cukes.info/}} and
RSpec\footnote{\url{http://rspec.info/}} were used as testing frameworks
for existing tests. We evaluated these frameworks, as well as considered
new frameworks in the beginning of the case study.\\

\subsubsection{Cucumber}

We worked with Cucumber during the first part of the case study, since
the major part of all existing tests was written using this framework.
Cucumber is a framework for acceptance-level testing using BDD
methodology. Tests, called \emph{features}, are written using an
ubiquitous language as seen in listing \ref{lst:cucumber_feature}. The
action for each line, called \emph{step}, of the feature is specified
using a \emph{step definition}, as seen in listing
\ref{lst:cucumber_step}.\cite{web:cucumber}\\

One benefit from using Cucumber is that all steps are reusable, which
means that code duplication can be avoided. However, it can be hard to
write the steps in such way that they benefit from this, and it
sometimes also requires a lot of parameters to be passed in to each
step. Cucumber also provides code snippets for creating step
definitions, which evades some unnecessary work.\\

Apart from these benefits, we found Cucumber tiresome to work with. The
separation between features and step definitions makes it hard to get an
overview of the code executed during the test, and it is often hard to
find specific step definitions. We also experienced problems with the
mapping step definitions, where the generated step definition simply did
not match the written step. In cases where we want to use the same step
definition, but use slightly different language (such as the steps on
line 12 and line 14 in code listing \ref{lst:cucumber_feature}),
adjusting the regular expression to match both steps can be hard.\\

The chosen level of testing is another big issue. We felt that using
TDD-methodology was hard to do with system-level tests since they take
long time to execute and affects a much larger part of the software than
the part we are typically working with when implementing new
functionality. Considering these drawbacks, we decided to not continue
the use of Cucumber for this thesis.\\


\begin{lstlisting}[caption=Example of a Cucumber test.,
                   label=lst:cucumber_feature, float=t, language=HTML]
Feature: creating new cookies
  As a bakery worker
  So that I can sell cookies to my customers
  I want to create a new cookie object

Background:
  Given a cookie type called "Chocolate chip"
  And I have created one cookie

Scenario: create a new cookie
  When I visit the page for creating cookies
  Then I should see 1 cookie
  When I create a new Chocolate chip cookie
  Then I should see 2 cookies
\end{lstlisting}


\begin{lstlisting}[caption=Cucumber step definition for the step on row 13 in code listing \ref{lst:cucumber_feature}.,
                   label=lst:cucumber_step, float=t]
When /^I create a new (.+) cookie$/ do |cookie_type|
  # Code for creating a new cookie
end
\end{lstlisting}


\subsubsection{RSpec}

Just like Cucumber, RSpec also claims to be made for use with
BDD\cite{web:rspec}. In contrast, it only uses descriptive strings
instead of a full ubiquitous language and can be used for writing
isolated unit-tests as well as integration- and browser tests.\\

We found it quite straightforward to write tests with RSpec, and to use
descriptive strings rather than function names for describing tests. The
plug-in rspec-mocks\footnote{\url{https://github.com/rspec/rspec-mocks}}
was used in some situations where mocking was required, since the RSpec
core does not include support for this. One major drawback of rspec-
mocks is that it allows stubbing non-existent methods and properties,
which as discussed in \fref{sec:theory_mocks} can be dangerous. We
worked around this by writing a helper for checking existence of
properties before stubbing them.\\

\begin{lstlisting}[caption=Example of RSpec tests for a module.,
                   label=lst:rspec, float=t]
describe Math do
  describe '#minus' do
    it 'returns the difference between two positive integers' do
      expect(Math::minus(3, 1)).to eq(2)
    end

    it 'returns the sum if the second integer is negative' do
      expect(Math::minus(5, -2)).to eq(7)
    end
  end

  describe '#plus' do
    it 'returns the sum of two positive integers' do
      expect(Math::plus(1, 2)).to eq(3)
    end
  end
end
\end{lstlisting}


\subsubsection{Factory girl}
\MakeShortVerb{\|}

One important framework used was
factory\_girl\footnote{\url{https://github.com/thoughtbot/factory\_girl}},
which is used for generating factory objects. As discussed in
\fref{sec:theory_mocks}, factory objects behaves just like instances of
model objects, but has several advantages. As alternatives to
factory\_girl, we also considered
Machinist\footnote{\url{https://github.com/notahat/machinist}} and
Fabrication\footnote{\url{http://www.fabricationgem.org/}}. Machinist
was discarded since it is no longer actively developed. We did not find
any significant differences between Fabrication and factory\_girl, and
therefore chose the latter since it was more popular.\\

There are many advantages of using a framework such as factory\_girl
rather than just instantiate model objects by hand (i.e. just write
|MyModel.new| in Ruby to create a new model instance). First of all,
factory\_girl automatically passes in default values for required
parameters, so that we only need to supply the attributes needed in a
particular test. Secondly, related objects are also created
automatically, which typically saves as huge amount of work compared to
manual creation of objects. Code listing \ref{lst:factory_def} and
\ref{lst:factory_use} shows a factory definition and its usage. The name
of the cookie and a new Bakery-object is created automatically since we
do not give them as parameters when using the factory.\\

We did initially have some issues with the creation of related objects
since the factory\_girl documentation did not cover working with
document-based databases such as MongoDB in our case, but we eventually
found out how to do this.\\

One feature that we felt was missing in factory\_girl was the ability to
specify attributes on related objects, for example to specify the name
of a Bakery when creating a new Cookie. It is of course possible to
first create a Bakery object and then creating a Cookie and pass in the
created bakery object, but a shortcut for doing this would have been
convenient in some situations. To our knowledge, Fabrication also lacks
this feature.\\

\begin{lstlisting}[caption=Example usage of the factory defined in code listing \ref{lst:factory_def}.,
                   label=lst:factory_use, float=t]
FactoryGirl.create :cookie, diameter: 4.5, thickness: 0.5
\end{lstlisting}

\begin{lstlisting}[caption=A factory definition for a Cookie model.,
                   label=lst:factory_def, float=t]
FactoryGirl.define do
  factory :cookie do
    name      'Vanlilla dream'
    diameter  1
    thickness 2
    bakery    { FactoryGirl.build(:bakery) }
  end
end
\end{lstlisting}



\subsubsection{Other tools}

TimeCop\footnote{\url{https://github.com/travisjeffery/timecop}} was
used in order to mock date and time for a test that was dependent on the
current date and time. We do not have much experience from using this
tool, but it worked as we expected for our specific use.\\


\subsection{Frameworks for browser-testing}
Capybara and site\_prism.


\subsection{Javascript test frameworks}

\label{sec:js_test}

There are a few different frameworks for testing Javascript or
CoffeeScript code. We had previously good experiences from working with
Jasmine\footnote{\url{http://jasmine.github.io/}}. We also found that
this framework seemed to be very popular, actively developed and had
good documentation\footnote{It is worth to mention that the Jasmine
documentation is basically its own test suite with some additional
comments. This works incredibly good in this case, presumably since it
is a testing framework and the tests are very well-written.}. Code
listing \ref{lst:jasmine} shows an example of a Jasmine test written in
CoffeeScript. The syntax of Jasmine is inspired by RSpec, which is
another advantage since RSpec is used for the server side tests.\\

\subsubsection{Test runner}

The Jasmine framework provides a way of writing tests, but a test runner
is also required in order to run the tests. Jasmine ships with a basic
test runner, which was used initially and worked natively using
the Jasmine Ruby gem. A screenshot of this test runner is shown in
\fref{fig:jasmine_runner}.\\

\begin{figure}
\centering
\includegraphics[width=0.8\textwidth]{results/choices/jasmine_runner}
\caption{The test runner bundled with Jasmine.}
\label{fig:jasmine_runner}
\end{figure}

The Jasmine test runner did however have multiple issues. First of all,
it runs completely in the browser. Switching to the browser and reload
the page in order to run the tests are not excessively problematic, but
might be tiresome in the long run when using a test-driven development
methodology.\\

A larger problem with the default test runner was that the asset
handling, i.e. the process of compiling CoffeeScript into Javascript.
Rails handles this compilation upon each reload when using the Jasmine
gem. Since the used version of Rails re-compiles all assets upon page
load if any file has been changed, this process takes a while, which
means that each test run could take up to 10-15 seconds even though the
actual tests only takes a fraction of a second to run. The asset
compilation also got stuck for apparently no reason once in a while.
Since the server port on which the test runner runs cannot be specified,
it is also impossible to restart the test runner without manually
killing its system process.\\

Another issue with the Jasmine test runner is that syntax errors are
printed in the terminal rather than in the browser window where the test
result is reported, which is confusing. In practice, syntax errors were
often undetected for a long time, which required more debugging than
necessary.\\

Due to these issues, we switched to the Karma\footnote{\url{http
://karma-runner.github.io/}} test runner. Karma originates from a
master's thesis by \citet{article:karma}, which covers several problems
with the Jasmine test runner as well as with some other Javascript test
runners. Karma was designed to solve several of these issues and to be
used with test-driven software methodologies. Tests are run in a browser
as soon as a file is changed, and the results are reported back to the
terminal and displayed as shown in \fref{fig:karma_runner}. Re-
compilation of source files and tests is very fast and we did not
experience any stability issues. One minor drawback is however that
pending tests is not displayed very clearly. This can be solved by using
a different Karma test reporter.\\

It took some effort getting Karma to work with Rails, since Karma is
written in Node.js and does not have any knowledge about which
CoffeeScript files that exist in the Rails application. The task of
finding the location of all such files became more complex since
external Javascript libraries such as jQuery was loaded as Ruby gems,
and therefore not even located inside the project folder. We ended up
using a slightly modified version of a Rake script from a blog entry by
\citet{web:saunier_angular} for bootstrapping Karma in a Rails
environment. This script basically collects a list of filenames for all
Javascript assets by using Rails internal modules for asset handling,
and injects it into Karma's configuration file.\\

After the case study, we discovered another Javascript test runner
called Teaspoon\footnote{\url{https://github.com/modeset/teaspoon}},
which is built for Rails and can discover assets by default. We believe
that this test runner looks very promising and may be more suitable for
Rails projects than Karma. At the time of this writing, Teaspoon does
not have support for the most recent version of Jasmine and therefore
does not work with our test suite. We were thus unable to evaluate
Teaspoon any further.\\

\begin{lstlisting}[caption=Example of Jasmine tests for a module (compare with code listing \ref{lst:rspec}).,
                   label=lst:jasmine, float=t, language=HTML]
describe 'Math', ->
  describe 'minus', ->
    it 'returns the difference between two positive integers', ->
      expect(Math.minus(3, 1)).toEqual(2)

    it 'returns the sum if the second integer is negative', ->
      expect(Math.minus(5, -2)).toEqual(7)

  describe 'plus', ->
    it 'returns the sum of two positive integers', ->
      expect(Math.plus(1, 2)).toEqual(3)
\end{lstlisting}

\begin{figure}
\centering
\includegraphics[width=0.8\textwidth]{results/choices/karma_runner}
\caption{The Karma test runner.}
\label{fig:karma_runner}
\end{figure}


\subsection{Test coverage}
\label{sec:results_coverage}

A quick overview of the results are presented in
\fref{tab:unit_coverage}. More information is presented in the following
sections. As discussed in \fref{sec:coverage_frameworks}, we were unable
to find any tool for analyzing anything else than statement test
coverage for Ruby.\\

\begin{table}[t]
    \centering
    \begin{tabular}{l l l}
        Phase & No. of tests & Test coverage\\
        \hline
        Before case-study &       59 & 42.24 \%\\
        After first part  &       58 & 55.25 \%\\
        After second part &       81 & 62.34 \%\\
    \end{tabular}
    \caption{ Statement test coverage of RSpec unit- and integration tests at different phases. }
    \label{tab:unit_coverage}
\end{table}

\begin{table}[t]
    \centering
    \begin{tabular}{l l l}
        Description & No. of tests & Test coverage\\
        \hline
        RSpec browser tests &     3 & 46.3 \%\\
        Cucumber browser tests &  8 & 61.2 \%\\
        All RSpec tests &        84 & 66.8 \%\\
    \end{tabular}
    \caption{ Statement test coverage including browser tests after the second part of the case study. }
    \label{tab:browser_coverage}
\end{table}

\subsubsection{Before the case study}

Before the start of this master's thesis, the average statement coverage
of all RSpec unit- and integration tests was 42 \%.\\

For the client-side code, no unit-tests existed. The test coverage was
thus zero at this state.\\


\subsubsection{After the first part of the case study}

The first part of the case study (described in \fref{sec:casestudy_1})
was focused on rewriting broken tests, since some of the existing tests
was not functional. During this period, a large part of the existing
test suite was either fixed or completely rewritten. Many of the
existing unit- and integration tests was rewritten and a few large
acceptance-level test was replaced by more fine-grained integration
tests. The statement coverage of the increased to 53 \% after this
part of the case study.\\

There were still no client-side unit-tests after this part, since the
focus was fixing the existing server-side tests. The client-side test
coverage was still zero at this state \\

\subsubsection{After the second part of the case study}
\label{sec:result_coverage_end}

The second part of the case study (described in \fref{sec:casestudy_2})
was focused on implementing new functionality while re-factoring old
parts as needed and write tests for new as well as re-factored code. The
first sprint of this part was focused on basic functionality, while the
second sprint was focused on extending and generalizing the new
functionality.\\

When the implementation of the new functionality was finished, statement
coverage of unit- and integration tests for the server side was 62 \%. A
subjective measure indicated that new and re-factored functions in
general has high statement coverage. In cases where full statement
coverage is not achieved, the reason is generally special cases. The
most common example is when error-messages are given when request
parameters are missing or invalid.\\

Statement test coverage for the client-side was 21 \% and the
corresponding branch coverage was 10 \%. A subjective measure
indicates that almost all of the newly implemented functions achieves
full statement coverage. The branch coverage is in general also high for
newly implemented functions, but conditionals for special cases (such as
when variables are zero or not set) are sometimes not covered.\\

\subsubsection{Test coverage for browser tests}

All metrics in the previous sections refer to test coverage for unit-
and integration tests. As one of the last steps of the second part of
the case study, browser tests was added in order to do system-level
testing.\\

The total statement test coverage for the RSpec browser tests alone was
46 \%, and the statement test coverage for all Ruby tests was 67 \%.
With the tools used, it was not possible to measure Javascript test
coverage for the test suite including browser tests.\\

The total statement test coverage for the Cucumber browser tests alone
was 61 \%. It was not possible to calculate the total coverage for
Cucumber and RSPec tests combined.\\


\subsection{Mutation analysis}
An alternative to draw conclusions from which paths of the code that is
run by a test, as done when using test coverage, is to draw conclusions
from what happens when we modify the code. The idea is that if the code
if incorrect, the test should fail. Thus, we can modify the code so it
becomes incorrect and then look at whether the test fails or not.\\

Mutation testing is done by creating several versions of the tested
code, and introduce slight modifications into each one of them. Each
such version containing a mutated version of the original source code is
called a \emph{mutant}. All tests which we want to evaluate are run for
each mutant. If at least one of the tests fails, the mutant is
considered to be \emph{killed}. We can measure the efficiency of the
test suite as the ratio of killed mutants versus the total number of
mutants.\cite{wiki:mutation}\\


\begin{lstlisting}[caption=Example of a piece of code before mutation,
                   label=lst:mutation_before, float=t]
    def odd?(x, y)
        (x % 2) && (y % 2)
    end
\end{lstlisting}


\begin{lstlisting}[caption=Two mutated versions of \ref{lst:mutation_before},
                   label=lst:mutant_1, float=t]
    def odd?(x, y)
        (x % 2) && (x % 2)
    end

    def odd?(x, y)
        (x % 2) || (y % 2)
    end
\end{lstlisting}

