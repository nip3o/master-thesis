\label{sec:choices_mutation}

There exist a few different tools for mutation analysis of Javascript
code. The ones we have found originate from academic research papers.
\citet{paper:mutandis} propose a solution that has been implemented as
a tool called Mutandis
\footnote{\url{https://github.com/saltlab/mutandis/}}.
\citet{paper:ajaxmutator} presents another approach that has been
released as AjaxMutator\footnote{\url{https://github.com/knishiura-
lab/AjaxMutator}}. \citet{paper:webmujava} propose a system-level
mutation testing approach called webMuJava.\\

Mutandis is based on website crawling tests. Although
\citeauthor{paper:mutandis} mentions that pure Javascript frameworks
have been tested using this tool, its implementation showed to be too
specific to be considered in our context. webMuJava does not seem to be
publicly available, and also seems to be too tightly integrated with a
specific back-end technique to be useful for Javascript-testing only.

AjaxMutator is in our opinion the most mature of all the considered
frameworks. It provides some basic documentation and installation
instructions, and the amount of implementation needed in order to begin
using it in a new project is reasonable. However, AjaxMutator currently
only supports Javascript tests written in Java using
JUnit\footnote{\url{http://junit.org/}}. Using it for mutation testing
tests written in CoffeeScript using Jasmine would probably be possible,
but the effort of doing this is outside the scope of this thesis.\\

For mutation analysis of Ruby code, we tried to use
Mutant\footnote{\url{https://github.com/mbj/mutant}}, which initially
seemed very promising since it is actively developed and is used in
several real-life software development projects. We were however unable
to get Mutant to work properly, probably due to our recent Ruby and
Rails versions. It was possible to start a mutation test, but the
command never finished and no result was presented. Unfortunately we
were unable to find any alternative tools for Ruby mutation testing.\\
